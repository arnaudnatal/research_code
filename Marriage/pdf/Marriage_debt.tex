\documentclass[a4paper, 11pt, onecolumn]{article} 

% arara: pdflatex 
% arara: bibtex
% arara: pdflatex
% arara: pdflatex
% arara: clean: {extensions: [ aux, bbl, out, toc, blg, thm ]}

%\usepackage[doi, natbibapa]{apacite}

\usepackage{enumitem}
\usepackage[english]{babel}
%if french
%\frenchbsetup{StandardLists=true}

\usepackage[T1]{fontenc}
\usepackage[utf8]{inputenc}

\usepackage{lmodern}

\let\CheckCommand\providecommand
\usepackage{microtype}
\usepackage{hyperref}
%\usepackage[pagebackref=true]{hyperref}
%\renewcommand*{\backrefalt}[4]{#1}

\usepackage{lscape}
\usepackage{graphicx}
\usepackage{amssymb,amsmath}
\usepackage{url}
\usepackage{longtable}
\usepackage{tabu}
\usepackage{siunitx}                        
%\usepackage{threeparttable} 
\usepackage{array}
\usepackage{booktabs}

%\usepackage[french]{authblk}
%\DeclareCaptionFormat{twodot}{:}
\usepackage[font=small,skip=1em]{caption}


\usepackage{setspace}
\usepackage{fullpage}
\usepackage{eso-pic}

\usepackage[explicit, clearempty]{titlesec}
\usepackage[tableposition=top]{caption}
%\usepackage{titlesec}
\usepackage[a4paper]{geometry}

\usepackage{adjustbox}
\usepackage{rotating}
\usepackage{hvfloat}
\usepackage{wrapfig}
\usepackage{tfrupee}  
%\usepackage{multicol}

\usepackage{calc}

\usepackage{lettrine}
\usepackage{oldgerm}

\usepackage{fancyhdr}
\usepackage{lipsum}  
\usepackage{lastpage}

\usepackage{changepage}

% *****************************************************************
% Annexes
% *****************************************************************
%\usepackage[title, titletoc]{appendix}
\usepackage[toc,page]{appendix}
%\renewcommand\appendixtocname{Annexes}
%\renewcommand\appendixname{Annxes}
%\renewcommand\appendixpagename{Annexes}

% *****************************************************************
% Estout related things
% *****************************************************************
\newcommand{\sym}[1]{\rlap{#1}}

\let\estinput=\input% define a new input command so that we can still flatten the document

\newcommand{\estwide}[3]{
		\vspace{.75ex}{
			\begin{tabular*}
			{\textwidth}{@{\hskip\tabcolsep\extracolsep\fill}l*{#2}{#3}}
			\toprule
			\estinput{#1}
			\bottomrule
			\addlinespace[.75ex]
			\end{tabular*}
			}
		}	

\newcommand{\estauto}[3]{
		\vspace{.75ex}{
			\begin{tabular}{l*{#2}{#3}}
			\toprule
			\estinput{#1}
			\bottomrule
			\addlinespace[.75ex]
			\end{tabular}
			}
		}

% Allow line breaks with \\ in specialcells
	\newcommand{\specialcell}[2][c]{%
	\begin{tabular}[#1]{@{}c@{}}#2\end{tabular}}

% *****************************************************************
% Custom subcaptions
% *****************************************************************
% Note/Source/Text after Tables
\newcommand{\figtext}[1]{
	\vspace{-1.9ex}
	\captionsetup{justification=justified,font=footnotesize}
	\caption*{\hspace{6pt}\hangindent=1.5em #1}
	}
\newcommand{\fignote}[1]{\figtext{\emph{Note:~}~#1}}

\newcommand{\figsource}[1]{\figtext{\emph{Source:~}~#1}}

% Add significance note with \starnote
\newcommand{\starnote}{\figtext{* p < 0.1, ** p < 0.05, *** p < 0.01. Standard errors in parentheses.}}

% *****************************************************************
% siunitx
% *****************************************************************
\usepackage{siunitx} % centering in tables
	\sisetup{
		detect-mode,
		tight-spacing		= true,
		group-digits		= false ,
		input-signs		= ,
		input-symbols		= ( ) [ ] - + *,
		input-open-uncertainty	= ,
		input-close-uncertainty	= ,
		table-align-text-post	= false
        }

% *****************************************************************
% Sources
% *****************************************************************
\newcommand{\sourcetab}[1]{\vspace{-1em} \caption*{ \textbf{Source}: {#1}} }
\newcommand{\sourcefig}[1]{\vspace{-2em} \caption*{ \textbf{Source}: {#1}} }

\addto\captionsenglish{\renewcommand{\figurename}{\textbf{Figure}}}
\addto\captionsenglish{\renewcommand{\tablename}{\textbf{Table}}}


% *****************************************************************
% Abstract
% *****************************************************************
\let\abstractname\abstracteng

% *****************************************************************
% Hypothèses
% *****************************************************************
\usepackage{ntheorem}
\theoremseparator{:}
\newtheorem{hyp}{Hypothesis}

% \makeatletter
% \newcounter{subhyp} 
% \let\savedc@hyp\c@hyp
% \newenvironment{subhyp}
 % {%
  % \setcounter{subhyp}{0}%
  % \stepcounter{hyp}%
  % \edef\saved@hyp{\thehyp}% Save the current value of hyp
  % \let\c@hyp\c@subhyp     % Now hyp is subhyp
  % \renewcommand{\thehyp}{\saved@hyp\alph{hyp}}%
 % }
 % {}
% \newcommand{\normhyp}{%
  % \let\c@hyp\savedc@hyp % revert to the old one
  % \renewcommand\thehyp{\arabic{hyp}}%
% } 
% \makeatother

% *****************************************************************
% Tableaux
% *****************************************************************
\addto\captionsfrench{\def\tablename{\textsc{Table}}}

% *****************************************************************
% Bigcenter
% *****************************************************************
%%% ----------debut de bigcenter.sty--------------
 
%%% nouvel environnement bigcenter
%%% pour centrer sur toute la page (sans overfull)
 
%\newskip\@bigflushglue \@bigflushglue = -100pt plus 1fil
 
%\def\bigcenter{\trivlist \bigcentering\item\relax}
%\def\bigcentering{\let\\\@centercr\rightskip\@bigflushglue%
%\leftskip\@bigflushglue
%\parindent\z@\parfillskip\z@skip}
%\def\endbigcenter{\endtrivlist}
 
%%% ----------fin de bigcenter.sty--------------
%%%% fin macro %%%%
\makeatletter
\newskip\@bigflushglue \@bigflushglue = -100pt plus 1fil
\def\bigcenter{\trivlist \bigcentering\item\relax}
\def\bigcentering{\let\\\@centercr\rightskip\@bigflushglue%
\leftskip\@bigflushglue
\parindent\z@\parfillskip\z@skip}
\def\endbigcenter{\endtrivlist}
\makeatother

% *****************************************************************
% Lignes de code
% *****************************************************************
\usepackage{listings}
\lstset{ 
basicstyle=\scriptsize\ttfamily,
breaklines=true,
keywordstyle=\bf \color{blue},
commentstyle=\color[gray]{0.5},
stringstyle=\color{red},
showstringspaces=false,
numbers=left,
numberstyle=\tiny \bf \color{blue},
stepnumber=1,
numbersep=10pt,
firstnumber=1,
numberfirstline=true,
frame=leftline,
xleftmargin=0.5cm
}
 
% *****************************************************************
% Auteurs en bleu
% *****************************************************************
%\renewcommand{\citep}[1]{\textcolor{teal}{\citep{#1}}}
%\renewcommand{\cite}[1]{\textcolor{teal}{\cite{#1}}}
\usepackage{xcolor}
\usepackage{colortbl}

\hypersetup{colorlinks,linkcolor={red},citecolor={teal},urlcolor={blue}}
%\newcommand{\ypenser}[1]{\textcolor[purple]{#1}}
\newcommand{\ypenser}[1]{\textbf{\color{purple}--#1--}}

% *****************************************************************
% Mail
% *****************************************************************
\newcommand{\email}[1]{\href{mailto:#1}{\nolinkurl{#1}}}

% *****************************************************************
% Résumé et mots clés
% *****************************************************************
 \newenvironment{resab}[1]
{\begin{adjustwidth}{0cm}{0cm} \hangafter =1\par
    {\normalsize\bfseries #1\ \\ }\normalsize}
{\end{adjustwidth}\medskip}

 \newenvironment{keywords}
{\begin{adjustwidth}{0cm}{0cm} \hangafter =1\par
    {\normalsize\itshape Keywords:}~\normalsize}
{\end{adjustwidth}\medskip}

 \newenvironment{jelcodes}
{\begin{adjustwidth}{0cm}{0cm} \hangafter =1\par
    {\normalsize\itshape JEL Codes:}~\normalsize}
{\end{adjustwidth}\medskip}

% *****************************************************************
% Poete
% *****************************************************************
\newcommand{\attrib}[1]{%
\nopagebreak{\raggedleft\footnotesize #1\par}}

% *****************************************************************
% Stata
% *****************************************************************
\newcommand{\Stata}{%
\textsc{Stata$^{\mbox{\scriptsize{\textregistered}}}$}
}

% *****************************************************************
% Encadré
% *****************************************************************
\usepackage{tikz}

% Thick
\def\checkmark{\tikz\fill[scale=0.4](0,.35) -- (.25,0) -- (1,.7) -- (.25,.15) -- cycle;} 

\newcommand{\titlebox}[2]{%
\tikzstyle{titlebox}=[rectangle,inner sep=10pt,inner ysep=10pt,draw]%
\tikzstyle{title}=[fill=white]%
%
\bigskip\noindent\begin{tikzpicture}
\node[titlebox] (box){%
    \begin{minipage}{0.94\textwidth}
#2
    \end{minipage}
};
%\draw (box.north west)--(box.north east);
\node[title] at (box.north) {#1};
\end{tikzpicture}\bigskip%
}

% *****************************************************************
% Changer la forme des titres
% *****************************************************************
 %\titleformat{\section}[block]			%section + style prédéfini par l'extension (block = 1 ligne)
 %{\sffamily\bfseries\LARGE\titlerule[1pt]}						%format pour le titre + label
 %{\sffamily\bfseries\LARGE}						%format pour le titre + label
 %{\sffamily\bfseries\LARGE\arabic{section}}		%format que pour le label
 %{0.5cm}								%espace qui sépare le label du titre
 %{#1}									%description du style pour le titre uniquement
 %\titlespacing{\section}				%section
 %{0cm}									%espace à gauche du titre
 %{3em}									%espace verticale AVANT le titre
 %{1em}									%espace verticale APRES le titre
 %%{0cm} 								%espace à droite du titre: mettre la même valeur que gauche pour un peu centrer

 %\titleformat{\subsection}{\sffamily\bfseries\Large}{\thesubsection}{0.4cm}{#1}
 %\titlespacing{\subsection}
 %{1cm}
 %{2em}
 %{1ex}

 %\titleformat{\subsubsection}{\sffamily\bfseries\large}{\thesubsubsection}{0.4cm}{#1}
 %\titlespacing{\subsubsection}
 %{2cm}
 %{2em}
 %{1ex}

% *****************************************************************
% Style de la date
% *****************************************************************
%\def\mydate{\leavevmode\hbox{\the\year-\twodigits\month}}
\def\mydate{\leavevmode\hbox{\the\month-\twodigits\year}}
\def\twodigits#1{\ifnum#1<10 0\fi\the#1}

% *****************************************************************
% En tête
% *****************************************************************

% Pour les numéros de pages
%\pagestyle{fancy}

% Si tu veux mettre les numéros genre: 1/22
% Il faut que tu écrives
% "\thepage/\pageref{LastPage}"
% Sans les " " dans les lignes en bas: \fancyhead[...

% Ca c'est pour enlever la barre horizontale sous l'entête
% Pour la laisser tu mets "1pt" au lieu de 0
\renewcommand{\headrulewidth}{0pt}
\renewcommand{\footrulewidth}{0pt}

% fancyhead pour l'entête
% fancyfoot pour le pied de page
% L=left; R=right; C=center
%\fancyhead[L]{\textcolor{gray}{\textsf{\textit{Revue de la littérature autour du mariage: le cas de l'Inde}}}}
%\fancyhead[R]{\textcolor{gray}{\textsf{Natal, A. (\mydate)}}}
%\fancyfoot[C]{\thepage}
%updmap.exe --admin

%\lhead{\textcolor{gray}{\textsf{\textit{Revue de la littérature autour du mariage: le cas de l'Inde}}}}
%\rhead{\textcolor{gray}{\textsf{Natal, A. (\mydate)}}}
\cfoot{\thepage}

% *****************************************************************
% Jatis
% *****************************************************************
\newcommand{\jati}[1]{\textit{j\={a}ti{#1}}}

% *****************************************************************
% Enlever le titre Table des matières
% *****************************************************************
%\makeatletter
%\renewcommand\tableofcontents{%
%    \@starttoc{toc}%
%}
%\makeatother

% *****************************************************************
% À développer
% *****************************************************************
\newcommand\dev[1]{\textbf{\textcolor{red}{#1}}}

% *****************************************************************
% Fonts
% *****************************************************************
%\usepackage{tgbonum}
\usepackage{kpfonts}

% *****************************************************************
% Taille des tableaux
% *****************************************************************
\let\oldtabular=\tabular
\def\tabular{\small\oldtabular}
%\def\tabular{\normalsize\oldtabular}

% *****************************************************************
% Style de la biblio
% *****************************************************************
\bibliographystyle{apacite}

% *****************************************************************
% Numérotation
% *****************************************************************
%\usepackage{lineno}

% *****************************************************************
% Titre et page de garde
% *****************************************************************
\usepackage{titling}
\setlength{\droptitle}{-2cm}
\pretitle{\begin{center}\fontsize{24pt}{10pt}\selectfont\bfseries}
\posttitle{\par\end{center}\vskip 1ex}
\preauthor{\begin{center}
    \large \lineskip 0.5em}
\postauthor{\par\end{center}}
%\thanksheadextra{1,}{}
\thanksheadextra{}{}
\setlength\thanksmarkwidth{.5em}
\setlength\thanksmargin{-\thanksmarkwidth}


% *****************************************************************
% Symboles
% *****************************************************************

\def\@fnsymbol#1{\ensuremath{\ifcase#1\or *\or \dagger\or \ddagger\or
   \mathsection\or \mathparagraph\or \|\or **\or \dagger\dagger
   \or \ddagger\ddagger \else\@ctrerr\fi}}
   
\makeatletter
\newcommand{\ssymbol}[1]{^{\@fnsymbol{#1}}}
\makeatother
   
   
   
% *****************************************************************
% Makecell
% *****************************************************************
\usepackage{makecell}
\newcommand\Tablenote[2]{\multicolumn{#1}{l}{\makecell[l]{\textit{Note:}~#2}}}


   
\usepackage[doi, natbibapa]{apacite}

\usepackage{enumitem}
\usepackage[english]{babel}
%if french
%\frenchbsetup{StandardLists=true}

\usepackage[T1]{fontenc}
\usepackage[utf8]{inputenc}

\usepackage{lmodern}

\let\CheckCommand\providecommand
\usepackage{microtype}
\usepackage{hyperref}
%\usepackage[pagebackref=true]{hyperref}
%\renewcommand*{\backrefalt}[4]{#1}

\usepackage{lscape}
\usepackage{graphicx}
\usepackage{amssymb,amsmath}
\usepackage{url}
\usepackage{longtable}
\usepackage{tabu}
\usepackage{siunitx}                        
%\usepackage{threeparttable} 
\usepackage{array}
\usepackage{booktabs}

%\usepackage[french]{authblk}
%\DeclareCaptionFormat{twodot}{:}
\usepackage[font=small,skip=1em]{caption}


\usepackage{setspace}
\usepackage{fullpage}
\usepackage{eso-pic}

\usepackage[explicit, clearempty]{titlesec}
\usepackage[tableposition=top]{caption}
%\usepackage{titlesec}
\usepackage[a4paper]{geometry}

\usepackage{adjustbox}
\usepackage{rotating}
\usepackage{hvfloat}
\usepackage{wrapfig}
\usepackage{tfrupee}  
%\usepackage{multicol}

\usepackage{calc}

\usepackage{lettrine}
\usepackage{oldgerm}

\usepackage{fancyhdr}
\usepackage{lipsum}  
\usepackage{lastpage}

\usepackage{changepage}

% *****************************************************************
% Annexes
% *****************************************************************
%\usepackage[title, titletoc]{appendix}
\usepackage[toc,page]{appendix}
%\renewcommand\appendixtocname{Annexes}
%\renewcommand\appendixname{Annxes}
%\renewcommand\appendixpagename{Annexes}

% *****************************************************************
% Estout related things
% *****************************************************************
\newcommand{\sym}[1]{\rlap{#1}}

\let\estinput=\input% define a new input command so that we can still flatten the document

\newcommand{\estwide}[3]{
		\vspace{.75ex}{
			\begin{tabular*}
			{\textwidth}{@{\hskip\tabcolsep\extracolsep\fill}l*{#2}{#3}}
			\toprule
			\estinput{#1}
			\bottomrule
			\addlinespace[.75ex]
			\end{tabular*}
			}
		}	

\newcommand{\estauto}[3]{
		\vspace{.75ex}{
			\begin{tabular}{l*{#2}{#3}}
			\toprule
			\estinput{#1}
			\bottomrule
			\addlinespace[.75ex]
			\end{tabular}
			}
		}

% Allow line breaks with \\ in specialcells
	\newcommand{\specialcell}[2][c]{%
	\begin{tabular}[#1]{@{}c@{}}#2\end{tabular}}

% *****************************************************************
% Custom subcaptions
% *****************************************************************
% Note/Source/Text after Tables
\newcommand{\figtext}[1]{
	\vspace{-1.9ex}
	\captionsetup{justification=justified,font=footnotesize}
	\caption*{\hspace{6pt}\hangindent=1.5em #1}
	}
\newcommand{\fignote}[1]{\figtext{\emph{Note:~}~#1}}

\newcommand{\figsource}[1]{\figtext{\emph{Source:~}~#1}}

% Add significance note with \starnote
\newcommand{\starnote}{\figtext{* p < 0.1, ** p < 0.05, *** p < 0.01. Standard errors in parentheses.}}

% *****************************************************************
% siunitx
% *****************************************************************
\usepackage{siunitx} % centering in tables
	\sisetup{
		detect-mode,
		tight-spacing		= true,
		group-digits		= false ,
		input-signs		= ,
		input-symbols		= ( ) [ ] - + *,
		input-open-uncertainty	= ,
		input-close-uncertainty	= ,
		table-align-text-post	= false
        }

% *****************************************************************
% Sources
% *****************************************************************
\newcommand{\sourcetab}[1]{\vspace{-1em} \caption*{ \textbf{Source}: {#1}} }
\newcommand{\sourcefig}[1]{\vspace{-2em} \caption*{ \textbf{Source}: {#1}} }

\addto\captionsenglish{\renewcommand{\figurename}{\textbf{Figure}}}
\addto\captionsenglish{\renewcommand{\tablename}{\textbf{Table}}}


% *****************************************************************
% Abstract
% *****************************************************************
\let\abstractname\abstracteng

% *****************************************************************
% Hypothèses
% *****************************************************************
\usepackage{ntheorem}
\theoremseparator{:}
\newtheorem{hyp}{Hypothesis}

% \makeatletter
% \newcounter{subhyp} 
% \let\savedc@hyp\c@hyp
% \newenvironment{subhyp}
 % {%
  % \setcounter{subhyp}{0}%
  % \stepcounter{hyp}%
  % \edef\saved@hyp{\thehyp}% Save the current value of hyp
  % \let\c@hyp\c@subhyp     % Now hyp is subhyp
  % \renewcommand{\thehyp}{\saved@hyp\alph{hyp}}%
 % }
 % {}
% \newcommand{\normhyp}{%
  % \let\c@hyp\savedc@hyp % revert to the old one
  % \renewcommand\thehyp{\arabic{hyp}}%
% } 
% \makeatother

% *****************************************************************
% Tableaux
% *****************************************************************
\addto\captionsfrench{\def\tablename{\textsc{Table}}}

% *****************************************************************
% Bigcenter
% *****************************************************************
%%% ----------debut de bigcenter.sty--------------
 
%%% nouvel environnement bigcenter
%%% pour centrer sur toute la page (sans overfull)
 
%\newskip\@bigflushglue \@bigflushglue = -100pt plus 1fil
 
%\def\bigcenter{\trivlist \bigcentering\item\relax}
%\def\bigcentering{\let\\\@centercr\rightskip\@bigflushglue%
%\leftskip\@bigflushglue
%\parindent\z@\parfillskip\z@skip}
%\def\endbigcenter{\endtrivlist}
 
%%% ----------fin de bigcenter.sty--------------
%%%% fin macro %%%%
\makeatletter
\newskip\@bigflushglue \@bigflushglue = -100pt plus 1fil
\def\bigcenter{\trivlist \bigcentering\item\relax}
\def\bigcentering{\let\\\@centercr\rightskip\@bigflushglue%
\leftskip\@bigflushglue
\parindent\z@\parfillskip\z@skip}
\def\endbigcenter{\endtrivlist}
\makeatother

% *****************************************************************
% Lignes de code
% *****************************************************************
\usepackage{listings}
\lstset{ 
basicstyle=\scriptsize\ttfamily,
breaklines=true,
keywordstyle=\bf \color{blue},
commentstyle=\color[gray]{0.5},
stringstyle=\color{red},
showstringspaces=false,
numbers=left,
numberstyle=\tiny \bf \color{blue},
stepnumber=1,
numbersep=10pt,
firstnumber=1,
numberfirstline=true,
frame=leftline,
xleftmargin=0.5cm
}
 
% *****************************************************************
% Auteurs en bleu
% *****************************************************************
%\renewcommand{\citep}[1]{\textcolor{teal}{\citep{#1}}}
%\renewcommand{\cite}[1]{\textcolor{teal}{\cite{#1}}}
\usepackage{xcolor}
\usepackage{colortbl}

\hypersetup{colorlinks,linkcolor={red},citecolor={teal},urlcolor={blue}}
%\newcommand{\ypenser}[1]{\textcolor[purple]{#1}}
\newcommand{\ypenser}[1]{\textbf{\color{purple}--#1--}}

% *****************************************************************
% Mail
% *****************************************************************
\newcommand{\email}[1]{\href{mailto:#1}{\nolinkurl{#1}}}

% *****************************************************************
% Résumé et mots clés
% *****************************************************************
 \newenvironment{resab}[1]
{\begin{adjustwidth}{0cm}{0cm} \hangafter =1\par
    {\normalsize\bfseries #1\ \\ }\normalsize}
{\end{adjustwidth}\medskip}

 \newenvironment{keywords}
{\begin{adjustwidth}{0cm}{0cm} \hangafter =1\par
    {\normalsize\itshape Keywords:}~\normalsize}
{\end{adjustwidth}\medskip}

 \newenvironment{jelcodes}
{\begin{adjustwidth}{0cm}{0cm} \hangafter =1\par
    {\normalsize\itshape JEL Codes:}~\normalsize}
{\end{adjustwidth}\medskip}

% *****************************************************************
% Poete
% *****************************************************************
\newcommand{\attrib}[1]{%
\nopagebreak{\raggedleft\footnotesize #1\par}}

% *****************************************************************
% Stata
% *****************************************************************
\newcommand{\Stata}{%
\textsc{Stata$^{\mbox{\scriptsize{\textregistered}}}$}
}

% *****************************************************************
% Encadré
% *****************************************************************
\usepackage{tikz}

% Thick
\def\checkmark{\tikz\fill[scale=0.4](0,.35) -- (.25,0) -- (1,.7) -- (.25,.15) -- cycle;} 

\newcommand{\titlebox}[2]{%
\tikzstyle{titlebox}=[rectangle,inner sep=10pt,inner ysep=10pt,draw]%
\tikzstyle{title}=[fill=white]%
%
\bigskip\noindent\begin{tikzpicture}
\node[titlebox] (box){%
    \begin{minipage}{0.94\textwidth}
#2
    \end{minipage}
};
%\draw (box.north west)--(box.north east);
\node[title] at (box.north) {#1};
\end{tikzpicture}\bigskip%
}

% *****************************************************************
% Changer la forme des titres
% *****************************************************************
 %\titleformat{\section}[block]			%section + style prédéfini par l'extension (block = 1 ligne)
 %{\sffamily\bfseries\LARGE\titlerule[1pt]}						%format pour le titre + label
 %{\sffamily\bfseries\LARGE}						%format pour le titre + label
 %{\sffamily\bfseries\LARGE\arabic{section}}		%format que pour le label
 %{0.5cm}								%espace qui sépare le label du titre
 %{#1}									%description du style pour le titre uniquement
 %\titlespacing{\section}				%section
 %{0cm}									%espace à gauche du titre
 %{3em}									%espace verticale AVANT le titre
 %{1em}									%espace verticale APRES le titre
 %%{0cm} 								%espace à droite du titre: mettre la même valeur que gauche pour un peu centrer

 %\titleformat{\subsection}{\sffamily\bfseries\Large}{\thesubsection}{0.4cm}{#1}
 %\titlespacing{\subsection}
 %{1cm}
 %{2em}
 %{1ex}

 %\titleformat{\subsubsection}{\sffamily\bfseries\large}{\thesubsubsection}{0.4cm}{#1}
 %\titlespacing{\subsubsection}
 %{2cm}
 %{2em}
 %{1ex}

% *****************************************************************
% Style de la date
% *****************************************************************
%\def\mydate{\leavevmode\hbox{\the\year-\twodigits\month}}
\def\mydate{\leavevmode\hbox{\the\month-\twodigits\year}}
\def\twodigits#1{\ifnum#1<10 0\fi\the#1}

% *****************************************************************
% En tête
% *****************************************************************

% Pour les numéros de pages
%\pagestyle{fancy}

% Si tu veux mettre les numéros genre: 1/22
% Il faut que tu écrives
% "\thepage/\pageref{LastPage}"
% Sans les " " dans les lignes en bas: \fancyhead[...

% Ca c'est pour enlever la barre horizontale sous l'entête
% Pour la laisser tu mets "1pt" au lieu de 0
\renewcommand{\headrulewidth}{0pt}
\renewcommand{\footrulewidth}{0pt}

% fancyhead pour l'entête
% fancyfoot pour le pied de page
% L=left; R=right; C=center
%\fancyhead[L]{\textcolor{gray}{\textsf{\textit{Revue de la littérature autour du mariage: le cas de l'Inde}}}}
%\fancyhead[R]{\textcolor{gray}{\textsf{Natal, A. (\mydate)}}}
%\fancyfoot[C]{\thepage}
%updmap.exe --admin

%\lhead{\textcolor{gray}{\textsf{\textit{Revue de la littérature autour du mariage: le cas de l'Inde}}}}
%\rhead{\textcolor{gray}{\textsf{Natal, A. (\mydate)}}}
\cfoot{\thepage}

% *****************************************************************
% Jatis
% *****************************************************************
\newcommand{\jati}[1]{\textit{j\={a}ti{#1}}}

% *****************************************************************
% Enlever le titre Table des matières
% *****************************************************************
%\makeatletter
%\renewcommand\tableofcontents{%
%    \@starttoc{toc}%
%}
%\makeatother

% *****************************************************************
% À développer
% *****************************************************************
\newcommand\dev[1]{\textbf{\textcolor{red}{#1}}}

% *****************************************************************
% Fonts
% *****************************************************************
%\usepackage{tgbonum}
\usepackage{kpfonts}

% *****************************************************************
% Taille des tableaux
% *****************************************************************
\let\oldtabular=\tabular
\def\tabular{\small\oldtabular}
%\def\tabular{\normalsize\oldtabular}

% *****************************************************************
% Style de la biblio
% *****************************************************************
\bibliographystyle{apacite}

% *****************************************************************
% Numérotation
% *****************************************************************
%\usepackage{lineno}

% *****************************************************************
% Titre et page de garde
% *****************************************************************
\usepackage{titling}
\setlength{\droptitle}{-2cm}
\pretitle{\begin{center}\fontsize{24pt}{10pt}\selectfont\bfseries}
\posttitle{\par\end{center}\vskip 1ex}
\preauthor{\begin{center}
    \large \lineskip 0.5em}
\postauthor{\par\end{center}}
%\thanksheadextra{1,}{}
\thanksheadextra{}{}
\setlength\thanksmarkwidth{.5em}
\setlength\thanksmargin{-\thanksmarkwidth}


% *****************************************************************
% Symboles
% *****************************************************************

\def\@fnsymbol#1{\ensuremath{\ifcase#1\or *\or \dagger\or \ddagger\or
   \mathsection\or \mathparagraph\or \|\or **\or \dagger\dagger
   \or \ddagger\ddagger \else\@ctrerr\fi}}
   
\makeatletter
\newcommand{\ssymbol}[1]{^{\@fnsymbol{#1}}}
\makeatother
   
   
   
% *****************************************************************
% Makecell
% *****************************************************************
\usepackage{makecell}
\newcommand\Tablenote[2]{\multicolumn{#1}{l}{\makecell[l]{\textit{Note:}~#2}}}


   


%\usepackage{fonttable}

\usepackage{import}

\usepackage[
singlelinecheck=false % <-- important
]{caption}

\usepackage{upgreek}

\newcommand{\ie}{i.e.}
\newcommand{\sd}{standard deviation}
\newcommand{\pp}{percentage points}
\newcommand{\aebe}{all else being equal}
\newcommand{\Aebe}{All else being equal}
\newcommand{\ote}{other things equal}
\newcommand{\Ote}{Other things equal}
\newcommand{\cl}{confidence level}
\newcommand{\lit}{\dev{literature}}
\newcommand{\PTCS}{PT\&CS}





% \renewcommand*\thetable{\Alph{section}.\arabic{table}}
% \renewcommand*\thefigure{\Alph{section}.\arabic{figure}}


% *****************************************************************
% Interlignes et marges
% *****************************************************************
\setstretch{1.25}
\geometry{%
left=2.5cm,
right=2.5cm,
top=2.5cm,
bottom=2.5cm,
%includefoot,
%headsep=1cm,
%footskip=1cm
}%

% *****************************************************************
% Page de garde
% *****************************************************************
\title{Marriage and debt in rural India}
\author{Arnaud Natal\thanks{Univ. Bordeaux, CNRS, GREThA, UMR 5113, F-33600 \textsc{Pessac, France} - \email{arnaud.natal@u-bordeaux.fr}}}
\date{\today}
\renewcommand\maketitlehookc{%
  \begin{center}
    %\textsuperscript{1}{\small Université de Bordeaux}
    \textsuperscript{}{\normalsize Very preliminary draft}
  \end{center}}

% ******************************************************************
\begin{document}
\maketitle

\hrule 
\vspace{0.3cm}

\begin{resab}{Abstract}
Research on marriage in developing countries has been somewhat narrow in scope because of both conceptual and data limitations.
While the feminist literature recognizes marriage as a key institutional site for the production and reproduction of gender hierarchies, little is known about the process through which this relationship operate.

\end{resab}

\begin{keywords}

\end{keywords}

\begin{jelcodes}

\end{jelcodes}

\hrule
%\linenumbers

\clearpage
\newpage
% ******************************
\section{Introduction}
% \addcontentsline{toc}{section}{Introduction}
\label{Introduction}
% ******************************

The abbot Jean-Antoine Dubois in his book \textit{Moeurs, institutions et cérémonies des peuples de l'Inde, Tome I} (Habits, institutions and ceremonies of peoples of India, volume I) of 1825 stated:
\begin{quote}
La plus grande affaire pour un Indien, la plus importante et la plus essentielle, celle dont on parle le plus, et à laquelle on se prépare de plus loin, est le mariage.
Un homme qui n’est pas marié est regardé comme étant sans état, et presque comme, un membre inutile de la société ; il n’est point consulté sur les affaires importantes ; on n’ose lui confier aucun emploi de quelque conséquence.
L’Indien qui devient veuf se retrouve placé dans la même position que le célibataire, et il s’empresse de se remarier bien vite.
\end{quote}

\cite{Regnault1891} il serait là un grand malheur qu’une jeune fille reste non mariée

\citep{Gupta1972}

\citep{Anukriti2018a}



	% ******************************
	\subsection{Marriage in economics}
	% ******************************

\cite{Grossbard1999a, Grossbard1999b} = 3 théories : marxiste, néoclassique avec la notion de marriage market et la théorie des jeux


\cite{Becker1973} : base pour parler de l'économie des marriages


	% ******************************
	\subsection{Marriage in India}
	% ******************************

\paragraph{Global characteristics of India}
Système des jati : la population est divisée en sous-groupe de caste localisé, des jatis, héréditaire et endogame, associées à un certain métier et occupant une position particulière dans la hiérarchie sociale qui sont exclusives : on ne peut appartenir qu’à un et un seul jati

\cite{Radhakrishnan1937} = mariage = sacrement central dans la vie des indiens
Allant jusquà l'alliance entre deux familles \citep{Sheela2003}

\cite{James2015} : stabilité de l'institution du mariage en Inde du Sud

En Inde, la majorité des familles sont élargies (nous parlons de joint family, voir de Hindu joint
family). Elles constituent à la fois une unité économique, religieuse et sociale \citep{Chandrasekhar1943}. Ainsi, dans cette optique, lorsque les frères grandissent et se marient, ils ne quittent
pas le foyer parental. Ce sont les épouses de ces derniers qui viennent habiter dans la maison
familiale.
les familles élargies, constituées du couple de parents, de leurs enfants et leurs parents
proches (grands-parents, oncles et tantes, cousins, etc.) qui vivent souvent à proximité;

Patriarcat
Dans ce système, les femmes sont valorisées
comme mères des garçons, ces derniers étant les hérités et prolongateurs des lignées familiales,
créant de grandes disparités de genre (à la différence des filles, le statut d’adulte est accordé
aux garçons dès leurs 15 ans) \citep{Marius2017}.

Importance de la religion:
Tout ce qui est rattaché à la famille, au mariage, divorce, à la pension alimentaire, l’adoption, l’héritage et la succession sont régis par la loi personnelle (ou privé) qui dépend de la communauté religieuse d’appartenance des individus, alors que tout ce qui dépend de l’espace public est uniforme pour tous les individus \citep{Moretti2004}.

Inde de Modi avec \textit{hindutva} qui favorise les rapports de domination et l'appartenance à la caste, ultra conservateur  \citep{Jaffrelot2018, DeVotta2002}
Plus généralement, les mandats de Narendra Modi
sont caractérisés par un retour en force des élites hindoues en reléguant au premier plan de la
politique les traditions hindouistes,
Donc de plus en plus conservateur le pays

Village council \citep{Blair2000, Alsop2000, Vijayalakshmi2003}

Grosso merdo deux types de mariage selon la caste :
En pratique, deux types de mariages existent selon le jati de l’époux \cite{Gupta1972}. D’une part,
chez les twice-born castes (les castes de brahmanes), le mariage est vu comme un acte irrévocable
et indissoluble renforcé par de nombreux rituels sacrés avant, pendant et après le mariage
\citep{Kapadia1966}. D’autre part, chez les nontwice-born castes (les autres j¯atis), le mariage est vu
comme un contrat civil et où il est toléré de se remarier (nata).

\paragraph{Arranged marriage}
En 2019 Naresh Tikait, chef du Bharatiya Kisan Union
(un clan d’agriculteur comptant plus de 20 millions de membres à travers l’Inde) et président
du Baliyan Khap des Jats (populations d’agriculteurs installés dans le nord-ouest de l’Inde et
au Pakistan et plus particulièrement dans le Pendjab et le Rajasthan) déclare que les femmes
n’ont pas leur mot à dire quant au choix de leur époux au vu du coût important que ces dernières
représentent pour leur famille respective. Pour reprendre ses termes: " Ça n’est pas correct.
Lorsqu’ils dépensent pour leur éducation, les parents devraient avoir leur mot à dire dans la
sélection du partenaire de vie de leurs filles. "


Mariage arrangé comme assurance chez les plus pauvres : migration assez "loin" pour limiter risques climatiques \cite{Rosenzweig1989} + transferts de fond de la part des femmes vers leur famille \cite{Belanger2011}

Chez les upper, assurance par la stimulation du réseau \cite{Munshi2009}

D'après \cite{Jejeebhoy2005} au Tamil Nadu, mariage avec migration de courte distance surtout pour le réseau

En Inde, choisir l’époux de sa fille est un des devoirs les plus importants pour un père, du
moins en Inde du Sud \cite{Jejeebhoy2005}

Le mariage sert à la reproduction, la socialisation et la transmission.
Théorie de l'alliance de Levi Strauss : La théorie de l’alliance (ou théorie structuraliste) s’intéresse aux relations entre les groupes. L’institution du mariage ne concerne donc pas uniquement deux individus, elle concerne deux groupes sociaux, deux familles, deux maisons, porteuse d’intérêts économiques, politiques et sociaux \cite{Héritier2015} 
Mariage = contrat d'alliance entre deux groupes

\citep{Rubio2014} pour des chiffres et on voit un progès, une émancipation \citep{Jejeebhoy2005}
Les lois suivent
Special Marriage Act et
l’Hindu Marriage Act. Ces dernières visent d’une part à autoriser le mariage inter religion et
inter caste et d’autre part autorisent la séparation et le divorce, à condition que les deux parties
soient d’accord. En 2010 un projet de loi a été porté au parlement afin de modifier ces lois en
proposant le divorce par " rupture irrémédiable ". Ainsi, toutes parties du mariage pourront
déposer une demande de divorce, sans forcément avoir obtenu au préalable l’accord de l’autre.
Ce type de mesure vise surtout à faciliter les procédures pour les femmes et leur offrir une
protection physique et juridique. Enfin, en 2019, la Cour Suprême indienne a décidé de mettre
en place dans les districts, des foyers sécurisés pour accueillir les couples inter religieux et inter
caste. Cette mesure fait suite au procès d’un crime d’honneur ayant eu lieu en 2018 (\url{http://toi.in/
XTMATb}).

\cite{Dharmalingam1994} : economics of marriage change in SI


\paragraph{Love marriage}
\cite{DeNeve2016} : economies of love in SI

\paragraph{Companionate marriage}
\cite{Fuller2008}, on the other hand, use the term
‘companionate marriages’ to refer to the rise of a type of marriage
among middle-class Tamil Vattima Brahmans in which couples try
to find out whether they will be congenial companions in marriage
without necessarily selecting each other or ‘falling in love’ before
marriage.21 This form of ‘companionate marriage’ and the courtship
that precedes it have become increasingly popular among the
wealthier Gounder families in Tiruppur too, but they are much less
prevalent among the city’s labouring classes.22 The romantic unions
and love marriages among Tiruppur’s working communities often flout
the ideal of caste endogamy, cross religious boundaries, and disregard
parental authority as well as differences of wealth and status.


\paragraph{Endogamy}
Au-delà du fait que c’est à la famille de choisir la partenaire, cette dernière doit appartenir,
en pratique à la même caste que le mari (endogamie), mais à un gotra (famille élargie), pravar
(lien religieux et spirituel), village et pinda (parenté) différent \cite{Radhakrishnan1937}.
Le mariage inter caste est autorisé lorsque c’est la femme qui appartient à une caste inférieure. Nous parlons alors
de mariage anumola qui représente en quelque sorte une ascendance sociale des femmes dalits \cite{Ahuja2015}. L’inverse (femme de caste supérieure au mari), appelé pratiloma, est formellement interdit \cite{Radhakrishnan1937}.

mariage inter caste est socialement condamnée par la société \citep{James2015}

\paragraph{Age at marriage}
Mariage selon son rang dans la famille, donc les premiers se marient tot \cite{Field2008}
Pour les filles mariage tot pour eviter avances sexuelles, pour virginité, filles = coutent cheres en éducation, santé, dot donc plus on marie tot plus on limite les dépensent et on réduit le montant de la dot \cite{Jensen2003}

\cite{Reddy1991} : structure familiale et age au mariage en Inde du Sud

Inde, l’âge médian au mariage est de 19 ans pour les femmes et 25
ans pour les hommes \cite{Paswan2016}.
Ces âges bas sont souvent associés avec un très faible niveau d’éducation, une mortalité
infantile élevée, un taux de violences domestiques important, ainsi qu’une forte probabilité de
divorce \cite{Field2008, Singh1996}.

Car
maximiser les chances que
cette dernière soit vierge, mais aussi à éviter les naissances hors mariage et les avances sexuelles
dont les jeunes femmes seules peuvent être victimes  \citep{Jensen2003}.
Rang de naissance donc nécessaire de s'y prendre tôt
Mariage plus jeune = dot moins élevée \cite{Jensen2003}
D’un point de vue plus légal, en 2007, le Parliament of India promulgue le Prohibition of
Child Marriage Act (aussi appelé Sarda Act) qui interdit le mariage des mineurs, pour tenter de
réduire, même à la marge, les mariages précoces.

\paragraph{Dowry}
\citep{Deolalikar1998} : La valeur monétaire de la dot peut représenter jusqu’à six fois le revenu
annuel de la famille de la future épouse et est bien souvent propice au surendettement, les
ménages ne disposant pas toujours de l’épargne nécessaire

Financement de la dot, épargne vs emprunt \cite{Caldwell1983, Corno2016}

Qu'est ce que la dot ? nature \cite{Caplan1984} + héritage ante mortem et 3 types de dot \cite{Tambiah1989} 

\cite{Dalmia2005} pas vraiment un héritage

Bride price \cite{Dalmia2005}

\cite{Srinivasan2005} : dowry practice change in SI

Différenciation des femmes sur le marché du mariage avec la dot \cite{Becker1991}

Montant: \cite{Dalmia2005}
Expliqué en partie par les castes \citep{Dalmia2005, Anderson2003}
Mais reprise de controle avec \cite{Jejeebhoy2005} et abolition en 1961

\cite{Bloch2004}
Often dowries are not perfectly observable in the community, and hence cannot be used as perfect sig-
nals of the groom's quality. Clearly, rumors may circulate about the amount of the dowry paid, but the exact
amount of the dowry is not usually public information. There are at least two reasons why dowries are not
publicly observed. Firstly, they are illegal and hence are usually not publicly announced. Secondly, there are
clear incentive problems in the revelation of dowries. Bride's families have an incentive to underreport
the amount paid, as high dowries may imply that the bride is of poor quality


\paragraph{Finance marriage}


\cite{Guerin2020c}
We attended the entire process for five of those, from the preparations, which involve
lively debates among family members and subtle calculations to determine the scale of the events, to
post-ceremonial remarks, both within the family (with heated debates on the generosity or conver-
sely the selfishness of the guests) and the neighbourhood (here with endless judgments on the greatness – or the smallness – of the event). Discussions about notebooks – a backbone of ceremonial
savings as we shall see below – were also key.7

First of all, these events are concrete oppor-
tunities to display and make visible the status (mariyātai) of the family. Ceremonies, insofar as they
gather the whole set of relations of individuals and their families, and insofar as they are in great part
funded by this same set of relations, express one’s mariyātai and contribute to building it. The scale of
the ceremony is evaluated in terms of guest numbers, their ‘quality,’ the quality of food provided, and
the gifts received: all this aims at maintaining, possibly uplifting or at least not downgrading the
organisers’ mariyātai (and that of their respective kin). Far from being simple, impersonal, individual
numbers of the kind found on a savings account statement, ceremonies reflect, crystalise and update
the social wealth of a group and its control over specific territories. These aspects are assessed in
terms of both the size of the local attendant population and the capacity to attract outsiders.

The scale of the ceremony mostly depends on two criteria. The scale of ceremonies recently
organised within the close circle (kinship and neighbourhood) is a first benchmark. Doing less
amounts to a downgrade in terms of honour. Doing better, even very slightly, is usually what is
expected.






\cite{Bloch2004} : 
A large proportion of marriage costs are in the form of dowries.
To an outsider these weddings can seem extremely lavish, especially in contrast
with the extreme poverty of rural Indian life, with large numbers of people invited for
feasts and ceremonies that can go on for several days. Such celebrations are, of
course, not unique to India but are of special relevance in very poor societies where
the money spent on weddings can be particularly wasteful given its high opportunity
cost. 

While there is a large economics literature on marriage markets following the
work of Becker (1990), and a rapidly expanding literature on dowries,1 wedding cel-
ebrations are driven by different considerations and have not been addressed at all by
economists, or much by other social scientists.

 Our in-depth
interviews indicated that dowries were largely driven by competition for scarce men
and by the quality of grooms. However, marriage celebrations were driven by differ-
ent criteria that had more to do with symbolic display than transfers, with the bride's
family primarily responsible for paying for the wedding, just as they were responsi-
ble for paying the dowry. The size of the celebration was usually justified as being
forced by norms in the community that are usually determined by observing the scale
of other recent weddings in the community.

In addition, however, to what was considered a minimally acceptable wedding,
many families tended to have particularly lavish celebrations influenced less by norms
in the village than by patterns in cities with celebrations by poor families imitating the
more extravagant patterns common in richer families. 
Comme dans le papier de \cite{Guerin2014a} où les HH tendent vers les modes de consommations urbains bien qu'ils soient ruraux et très pauve

When asked why he
had spent so much money on a wedding that was obviously well beyond his means, the
father said that his daughter had married into a "good family" and he wanted to have a
"show." It should be noted that the son-in-law's father was a relatively wealthy
landowner from another village. Such clear distinctions between the motives driving
dowries, which our respondents said were about "purchasing" desirable grooms, and
wedding celebrations, which they indicated were of more symbolic value, were
expressed by most of the individuals with whom we had in-depth interviews.

Clearly, wedding celebrations have a lot to do with social status and prestige. What
does status mean in this context and how does it matter? Anthropologists have long
believed that Indian concepts of individuality differ markedly from the Western. An
Indian is defined not just by his or her own accomplishments and character, but also
by their circle of acquaintances and friends-how many important people they know,
and the status and respect accorded to them by their social group. Mines (1994), in a
study of a South India community, shows that men will often describe themselves to
a stranger not simply by providing information about who they are and what they do,
but by listing all their prominent acquaintances. In our own fieldwork, one village
leader described himself to us in a similar manner, "I am not a big man, but my father
was a freedom fighter and my daughter is married to a big family in Patilur village
(about 100 miles away). They have lots of land and her father-in-law is a big Congress
politician in the area." Thus, the village leader's sense of self seemed to be derived
from the "big men" to whom he was related.

Furthermore, mobility within a village is often achieved by imitating the behaviors
of families of higher social orders (Srinivas 1989). Families devote a great deal of
effort and expense to the presentation of external attributes. Household decisions are
often made with an emphasis on how one's family will be viewed by others: What will
others say? What will they think? For the parents of a daughter, marriage is potentially
the most important source of mobility because marrying into a "good family" can
greatly enhance how a family is viewed by its peers, and a prestigious match is an
occasion for great celebration and status displays. This, more than anything, explains
why some weddings are particularly lavish. Status is a value in itself.
While it may
also generate some secondary benefits like greater access to networks and informa-
tion, families clearly gain direct Utility from simply moving up the social ladder and
being associated by marriage with a prestigious, wealthy pedigree.3 Thus, when a
family marries into a rich family it is in their interest to demonstrate this to the rest of
the village, particularly if the rest of the village does not know the new in-laws. The
most effective way of signaling a family's newfound affinity-derived status is to have
as lavish a wedding as they can possibly afford. On the flip side, if a family marries a
poor local family-well
known to everyone in the village, this may also be an occa-
sion for celebration-but lavish displays are no longer necessary because not much
can be gained by signaling.

Marriage celebrations, however, follow a different pattern. They are large, averag-
ing about four months income, though much smaller than the dowry. In about 25 per-
cent of the cases, the bride's family reported spending nothing on marriage
celebrations, which is indicative of a small function restricted to immediate family
members or of rare cases (more common in the past) where the groom's family cele-
brated the wedding. Excluding these families, marriage expenses average over 5,000
rupees, which is about a third of the annual income of an average family. While the
celebration costs are small relative to the dowry, they still represent a major burden
for families at a time when substantial resources have to be spent on the dowry.

. When the bride's family is marrying into
a family from another village, however, the quality of the match is unknown to the
village of the bride's family and this change in status can be signaled to the village
by the size of the wedding celebration. This is suggested by the fact that in the sam-
ple analyzed in this paper, the average distance from the wife's home village for
families that practice village exogamy is 34 kilometers, which can be a considerable
distance in a society where roads and public transportation are very poor. In addi-
tion, while families themselves are quite careful about obtaining reasonably good
information on their prospective in-laws, the new in-laws are often quite unknown
to the other families in the village. Therefore, variations in village exogamy provide
a natural experiment that allows us to test if wedding celebrations are driven by sig-
naling motives

The processes by which dowries and marriage expenses are determined are some-
what different. Dowries are almost always the result of direct negotiations between
the families of the bride and the groom, though there is clear sense of what a reason-
able dowry is for any given match which is a function of the marriage market.
Wedding celebrations are also, to a degree, the result of negotiations between the
two families but to a much lesser extent than the dowry. A bride's family often spends
well beyond the level expected by the groom's family as a result of its own set of
incentives







	% ******************************
	\subsection{Political aspects and questionning}
	% ******************************

\paragraph{Political motivations}
depuis quelques années, la Banque Mondiale s'attelle à reconnaitre et comprendre l'importance des traditions dans le développement économique \citep{Banque2015, Collier2017}. Cependant, le manque de données lié, entre autres, à la difficulté de la collecte (donnèes fiables difficiles à obtenir) nous amène à une compréhension relativement limitée du rôle de la culture et des traditions dans les politiques de développement \citep{Ashraf2018}. Comme le soulignent \cite{Ashraf2018} "il est de plus en plus reconnu que des stratégies uniformes [en matière de politiques de développement économique] ne sont peut-être pas optimales et que la compréhension du contexte culturel d'une société peut présenter des avantages lors de la conception de ces dernières" \citep{Rao2004, Banque2015}.



\paragraph{COVID-19}

\cite{Guerin2020} : conséquences debt

\cite{Shahare2021} : conséquences migrants 

lockdow affected the lives of 66 million migrant daily wage workers
(Census of India, 2011). Almost 77 percent of the workforce in India is included under
this vulnerable employment category (ILO, 2019). They are the main pillars of nation-
building of the Indian economy and contribute 50 percent to GDP (National
Commission for Enterprises in the Unorganised Sector (NCEUS), 2007). 
90 percent of the migrant workers belong to lower castes,
tribes,1 and religious minorities in the rural areas (Desai \& Vanneman, 2011–2012;
R. Srivastava, 2020).


\cite{Desai2021} : conséquences en emploi par genre

Estimates based on random-effects
logistic regression models show that for men, the predicted probability of
employment declined from 0.88 to 0.57, while for women it fell from 0.34 to
0.22. Women’s concentration in self-employment may be one reason why their
employment was somewhat protected. However, when looking only at wage
workers, the study finds that women experienced greater job losses than men
with predicted probability of employment declining by 72 percent for women
compared to 40 percent for men. The findings highlight the gendered impacts
of macro crises and inform policy considerations through ongoing phases of
lockdowns and relaxation.



\paragraph{Questionning}

Most of the literature on marriage in India deal with dowry and the consequences in terms of empowerment \citep{Roy2015, Alfano2017, Srinivasan2007} or the age at marriage and the child marriage \citep{Vogl2013, Sheela2003}
Few economist deal with the finance of marriage.
Recently, \cite{Anukriti2020} investigate the chanels to finance dowry in India.
They show that actual saving is drived by the expected amount of dowry [method].
\cite{Bloch2004} is one of the only research article that try to investigate the determinants of marriage expenses [literature].

\cite{Guerin2020c} show that ceremonials (including marriage) is a form of savings in rural India.
Married expect withdraw a profit from marriage with gift.

We try to understand the determinants of marriage benefits after 

\cite{Krishna2003, Krishna2004} : debt as a source of poverty 

\cite{Guerin2014a} : the problem is not debt, but to whom one becomes indebted

Relation between ceremonies fees, dowry and gift

Est-ce que je ne peux pas régresser le résidu du coût du mariage sur la dot ou un truc du genre ?

\begin{enumerate}
\item Est-ce que plus on dépense dans le mariage, plus on a des retours sur investissement ?
\item Est-ce que la caste, le type de mariage, etc. déterminent les marriages fees ?
\item Quels sont les déterminants de la dot ?
\end{enumerate}

D'après Isabelle, la dot joue un peu le rôle d'aspiration, est-ce que si je prends la dot net des facteurs classiques je n'ai pas le niveau d'aspiration ?
Je peux potientiellement regarder l'impact de ces aspirations sur qql chose ?




% ******************************
\section{Data and methodology}
% ******************************

	% ******************************
	\subsection{Data}
	\label{subsection:data}
	% ******************************

Our empirical analysis is based on NEEMSIS-1 \& NEEMSIS-2 (Networks, Employment, dEbt, Mobilities and Skills in India Survey) surveys carried out respectively in 2016-17, and 2020-21 \citep{NEEMSISreport, NEEMSIS2017}.
These surveys are the second and third waves of a longitudinal data collection project\footnote{Project took place within two broader research programmes located within the Observatory of Rural Dynamics and Inequalities (\url{https://odriis.hypotheses.org/}) at the French Institute of Pondicherry, India.} start in 2010 with RUME (RUral Microfinance and Employment survey) project in ten villages of Tamil Nadu.
Located in the Cuddalore and Villupuram districts, a mostly agricultural area, economies benefits from the proximity of two large industrial towns (Neyveli and Cuddalore) and a regional business center (Panruti).

RUME randomly selected 405 households using stratified sample framework based on three dimensions: proximity to small towns (Panruti, Villupuram and Cuddalore), an agro-ecological criterion, and caste affiliation.
Thus, half of villages have irrigated land (the other half is dry) and within villages, half of the sample was selected from the mostly upper and middle caste part of the village (Ur) while the other half from the Colony part, where dalits (the ex-untouchables) mainly live. 
NEEMSIS-1 recovered 388 households (4.19\% attrition rate) and randomly selected 104 news households (for a total of 492 households) from these 10 villages, based on the same method. 
%Given that some households had migrated elsewhere between the 2010 and 2016-17 sampling periods (13\% of the recovered households).
NEEMSIS-2 recovered 485 households (1.42\% attrition rate) from 2016-17 and recovered 10 households from 2010 that were not recovered in 2016-17.
Moreover, 100 news households were randomly selected (for a total of 595 households).

While RUME survey focus on employment, migration and remittances, financial practices (such as loans, savings, lending practices, gold), agriculture, consumption and housing at household level, a new individual questionnaire have been added to NEEMSIS.


NEEMSIS-1 \& NEEMSIS-2 surveys stand out from other Indian data sources such as the All India Debt and Investment Survey (AIDIS), as it has the rare and valuable advantage of recording debt at the individual level (identifying the person who went to the lender and borrowed in her own name).

Regarding the reliability, the great expertise of the team\footnote{Some members of the research team are present since more than 20 years on the region for numerous quantitative and qualitative surveys.} helped to formulate questions appropriately.
This for instance involved using particular terms that are less degrading than the generic term ``debt'', lists of the main local lenders, and asking indirect questions.
As stated by \cite{Reboul2021} who used the same dataset, ``data accuracy is [...] reflected by an incidence of indebtedness found higher than in the estimates of the nationwide AIDIS: 99\% of households are in debt in our case study, as opposed to 30\% in rural Tamil Nadu in 2012 according to the AIDIS \citep{NSSO2014}.'' 
Moreover, the moderate magnitude of the survey, compared to nationally representative datasets, ensures the high quality of the data and the tablet-based mode of data collection improved data quality in including constraints on answers to prevent inconsistencies. 



	
	
	
	% ******************************
	\subsection{Methodology}
	% ******************************

Statistiques descriptives sur les mariages


Based on our ethnographic findings, we follow a "participatory econometric" \cite{Rao1997, Rao2002}.







% ******************************
\section{Real cost of marriage}
% ******************************


NEEMSIS 2 data at marriage level for description (n=117)
\begin{itemize}[leftmargin=*]
\item 85.47\% of our marriages are arranged and only 6.84\% are consanguineous;
\item We find strong difference between caste for arranged marriage: 69.23\% for upper caste and 91.43 for middle;
\item In 85.47\% of marriage, the economic situation of husband or wife is the same than the observed husband or wife;
\item On average, the number of people at the wedding increase with the caste and on all sample, 372 persons are present at a marriage;
\item 90\% of marriage take place before lockdown and 8.55\% after;
\item We have 2 marriage during the lockdown;
\item 83.75\% of marriage take place in the same caste (dalits, middle, upper);
\item [...] thus 16.25\% (which represent 19 marriages) are inter caste\footnote{As \cite{Guerin2014a} stated, avec la dette normalement ils ne prêtent pas entre eux.} [literature] which is supported by the government;
\item 
\end{itemize}


NEEMSIS 2 data at marriage level (n=117)
\begin{itemize}[leftmargin=*]
\item Marriage cost, engagement cost and total cost increase with caste;
\item 50\% of marriage cost and engagement cost is provided by husband and 50\% by female for 50\% of marriages (44\% by wife for marriage on average and 74\% by wife for engagement on average);
\item Dowry represent 54\% of wife total cost (including engagement, marriage and dowry) but dowry represent 176\% of marriage and engagement fees;
\item Wife provide, on average, 66\% of total cost of \textbf{union (including engagement, marriage and dowry)} and the median is at 69\%;
\end{itemize}



NEEMSIS 2 data at individual level (male=59; female=58)
\begin{itemize}[leftmargin=*]
\item In our sample, the marriage cost represent 47\% of the monetary value of assets;
\item [...], the engagement represent 11\% of the monetary value of assets;
\item For our sub sample of female, dowry represent 69\% of the monetary value of assets (the median is at 43\%);
\item Relative to the annual labor income of the household, the cost of marriage represent 85\% for male and 166\% for female;
\item The dowry represent 235\% of the annual labor income, around 2 years and 4 months of labor income;
\item The total cost of union represent 1 year of income for male on average and more than 3 years for female;
\item If we looked the expenses, they represent, on average, 85\% of the cost for male and 59\% of the marriage cost for female  (not including dowry, only marriage fees) which suppose that female need to borrow more than male to finance the marriage fees;
\item Surprisingly, when the husband/wife family is in better economic situation than the observed, the proportion of the payment is not higher than when the husband/wife family is in the same situation;
\item [...] Indeed, when the observed is a male (husband) and the wife's family is in better situation, the ratio of $\frac{marriage wife cost}{marriage total cost}$ equal 0.40 on average while it is at 0.46 on average when the wife family is in the same situation (for observed wife, the ratio is respectively 0.49 and 0.56);

\end{itemize}

voici une ref à 

NEEMSIS 2 data for gift (male=59; female=58)
\begin{itemize}[leftmargin=*]
\item 95\% of individuals (husband or wife of our sample) received gift; 
\item On average, individuals receive 133 000 rupees as gift for their marriage;
\item [...] that represent 38\% of the monetary value of assets and around one year of labor income (98\%);
\item Gift represent 138\% of the cost provided by the individual on average (the median is at 90\%);
\item Thus, 42\% of the individuals have a net profit on marriage fees (gift amount higher than cost);
\item This proportion is higher for wife than for husband because they pay less on marriage fees;
\item If we looked at the net benefits of marriage for wife ($benefits=gift-(fees+dowry)$) and husband ($benefits=(gift+dowry)-fees$), the average benefits is respectively at -266 120 rupees and +209 850 rupees;
\item [...] that represent a gain of 38\% of their assets for male (median) and a loss of 42\% for female (median);
\item in terms if labor income, it represent a gain of 11 months of labor income for husband family and a loss of 33 months for wife family;
\item Again with income, the median is at 0.61 for husband, which represent more than 7 months and the median is at 1.1 for wife, which represent more than 13 months;
\end{itemize}



\cite{Guerin2014a} : Honouring reciprocity in ceremonies has always been a source of permanent pressure.
Many interviewees make clear that they prefer going into debt outside the family circle
to meet their own needs. This is a matter of freedom, as kin support calls for constant
justification (niyayapadthanum). Some say they borrow from their kin only for
"justified" reasons, which are mainly ceremony, housing and health costs. The
obligation of reciprocity (tiruppu) is also a burden. Not only should the debt be repaid,
but the debtor should be able to lend in return when the creditor is in need. 



\cite{Guerin2020c} Weddings are the most important events in terms of amount,
which vary according to social group. For Dalits, typical amounts in the region now range from three
to six lacks (300,000 to 600,000 INR), which on average amounts to four to eight years of household
income. These amounts have risen considerably over recent decades and include the cost of the cer-
ebration, gifts to close relatives and the dowry.

At the same time, one is supposed to organise an event within one’s financial and
human resources. Human resources are needed to help with the preparation of the food, service
and cleaning – the events usually bring together several hundred guests who must be welcomed, fed and sometimes accommodated. Financial resources depend on available savings (has the family
been able to accumulate gold, possibly land that can be offered, sold or pledged?), borrowing
capacities (how much can the family borrow and what is its creditworthiness in the eyes of potential
creditors?). Last but not least, how much can be expected from the guests in gifts (moi)? 
Le moi panam (la somme des dons offerts lors de l'événement) représente le plus souvent une part importante des dépenses de cérémonie, voire plus. En fait, mis à part les mariages de filles qui conduisent toujours à une «perte» due à la dot, chaque cérémonie devrait conduire à un surplus, un «profit», comme on nous l’a souvent dit (le terme anglais est fréquemment utilisé). En fait, le moi panam est le «nœud» de l’événement, son pilier. En d'autres termes, les cérémonies sont une sorte de pari sur la générosité de leur cercle social (à la fois en argent et en temps), qui à son tour dépend des obligations que les organisateurs ont accumulées au fil des ans. Pour les funérailles, les invités doivent fournir du riz et des articles de cuisine, qui doivent ensuite être retournés en temps opportun.

As we were told once, each event ‘is a link
to the past, the present and the future.’ It is an opportunity to show ‘the strength of your family
and relatives’ as we were also told, but the strength in question is a process: each event updates
prior relations and prefigures upcoming relations.


The event cost them around 1,50,000 INR10 (mostly food and transport, which was taken care of
for distant guests). They received 1,92,365 INR as gift (moi panam), which means a ‘surplus’ of
around 40,000 INR, that Sivaselvi used to pay off past debts. Gifts in cash ranged from 50 to 3000
INR. Gifts in gold ranged from a quarter of sovereign to 2.5 sovereigns (from around 5500 to
55,000 INR). The number and profile of the guests was a great source of satisfaction: most of the
neigbhours came, and many from outside and sometime long distances, revealing the territorial
spread of the family. In terms of gifts, however, Sivaselvi explains bitterly that she was expecting
a surplus of twice more.











% ******************************
\section{Marriage debt: burden of good debt}
% ******************************

Après avoir balancé mes chiffres, dire que c'est de la bonne dette d'un point de vu sociale

Je dois revenir sur le fait que ce sont les plus pauvres qui s'endettent pour le plus pour les cérémonies. De plus, c'est pour cela que nous trouvons les montants de dettes les plus élevés.
Je devrais faire un panel avec les prêts pour montrer que les montants moyens de prêts pour mariage ne font que augmenter et que ce sont les castes les plus basses qui s'endettent avec ça.
Faire les ratios panel dette mariage / revenus annuel du travail pour montrer cette augmentation.


\cite{Guerin2012} : Debt for why?



\cite{Guerin2014a} : social meaning of debt
While “financial inclusion” policies are now central to the political agendas of Indian public
policy makers (Garikipati 2008), private stakeholders such as NGOs and banks (Srinivasan
2009), and international organisations (World Bank 2007), this concern remains extremely
pressing

Over indebtedness and unmanageable can see as disaster : 
Dette --> suicide : Mohanty B. B. We are Like the Living Dead’: Farmer Suicides in Maharashtra, Western India, The Journal
of Peasant Studies, 32(2): 243–276, 2005

But it is also daily indebtedness: not necessarily lead to the dramatic situations observed amougst cotton farmers or microfinance clients but it can be nevertheless a source of impoverishment, pauperization and dependency \citep{Guerin2014a}

As for
the purposes for taking on debt, table 3 shows how in terms of debt size, the most significant
reasons include economic investment (mainly in agriculture) and ceremonies. In terms of the
number of loans, household expenses, economic investment and ceremonies are the most
common purposes. Here too significant disparities emerge as regards debt purpose: low castes, landless households and labourers more often borrow to cover daily survival costs and
ceremonies, while middle castes, landowners and producers more often borrow for economic
investment.

If we examine the main causes of over-indebtedness (see table 4 below), the most frequent are
ceremonies (42.65\% of households), housing and health (25\% and 23.53\%). These are
followed by failed economic investments (17.65\%); most frequently obtained for agricultural
purposes such as well digging or tractor purchase. 

When they are asked to talk about over-indebtedness, the borrowers rarely use
the amount of debt as an indicator. It is more the nature of the debt and the nature of the debt
relationship that determines whether debt is considered a burden or not.

Debt is a marker of social hierarchy in
kinship groups, the neighborhood and community alike. People try to avoid debts degrading
to their status, or at least try to pay back these debts first.

Borrowing from mobile lenders is seen as the most degrading practice, reserved for low castes
(and to women). Mobile lenders come to households’ doorsteps, precluding any form of
discretion. They do not request any collateral but use coercive enforcement methods. The
lenders themselves state that low caste individuals and women are more prepared to tolerate
abusive language from them.

The sense of debt as something immoral also depends upon the hierarchical positions of the
lender and the borrower. On the borrower’s side, the norm is to contract loans from someone
from an equal or higher caste." They do not take water from us, do you think they would take
money?" is something the low castes often said to us. On the creditor side, some upper castes
refuse to lend to castes who are too low in the hierarchy in comparison to them, stating that it
would be degrading for them to go and claim their due. To ask an upper caste whether he is
indebted to a lower caste can be considered as an insult.

The use of the term terinjavanga presupposes the idea of mutual
acquaintanceship (‘I know him/her, he/she knows me’)

While family support is crucial for
ceremonies and rituals

The most sensitive debts are those that do not respect the rules of rights and
obligations dictated by blood and alliances/bonds. For instance, borrowing money
from the bride’s kin is often a last resort, because it admits that the groom’s family is
unable to meet its responsibilities. Sometimes individuals may have no choice, but they
will be prompt in repayments, as the case study below illustrates.
Kathirvelu has an outstanding debt of 165 000 INR borrowed from twelve lenders, at rates ranging from 0 to
5\% per month. Of this, he considers only 28 000 INR as problematic. 20 000 INR is from pledging his
daughter-in-law’s jewellery. She allowed him to pledge 10 000 INR, but without her knowledge he
borrowed twice as much. The interest rate is rather low (1.5\% monthly) but this is not the problem: “I have
to keep face with the bride’s kin” he says. Another priority is a debt of 8000 INR from his son’s recruiter.
Here too, he does not want to show the bride’s kin that he is indebted. The family reputation is at stake: he
still has two daughters to marry, “we will never get them married if we have a reputation as indebted
family” (Kathirvelu, SC, agriculture coolie and tenant)

“Bad” debts are
rarely the most expensive, financially speaking, but those that tarnish the reputation of the
family and jeopardize its future, especially children’s marriages. Bad debts serve to reveal that a household is unable to maintain its position in the social hierarchy. The poor do
undertake financial reasoning, but financial criteria are not a priority and debt behaviours
stem from subtle arbitrations between financial costs and social status.
\cite{James2020} : debt, marriage and consumption





\cite{Bhukuth2018} : Poor households may take out credit for subsistence reasons. Yet they need credit not
only to buy consumer goods, but also to pay for health expenses, house repairs and social
investment (Krishna et al., 2003, 2004; Sriram \& Parhi, 2006; Bhukuth et al., 2007).
Therefore, servicing these debts becomes the main source of household poverty (Krishna
et al., 2003, 2004; Sriram \& Parhi, 2006).
Moreover, these households do not generally have access to the formal credit market.
Although the Indian government nationalised the banks in 1969 in a move to encourage
them to provide the rural population with formal credit, not all the people have benefited
(Basu, 2005). From the 1990s, liberalisation and privatisation result in credit rationing
among rural households, in particular for dalits (Chavan, 2007; Ramachandran \&
Swaminathan, 2005; Shah et al., 2007; Shetty, 2004). Poor households, most of whom
are landless, are denied access to the credit market because they have no savings or bank
account, which is a prerequisite to obtain credit from a bank.



\cite{Carswell2021}

The most commonly used source of credit is moneylenders: nearly 39\% of
households owed money to moneylenders. Moneylenders tend to lend fairly small amounts of money (average debt
amongst our sample to moneylenders being Rs 5,317) under what is known as kandu vatti, best translated as ‘metre
interest’ or ‘running interest’. Moneylenders usually belong to landed castes from nearby villages or towns. They
travel around on motorbikes and charge usurious interest rates of up to 10\% per month. Such loans require weekly
payments over 10 weeks. But moneylenders also loan larger amounts against collateral, such as a patta (land docu-
ment) or a vehicle registration document. For such loans, families typically approach a moneylender in town and tend
to repay a monthly sum that contains both interest and capital repayment.

The second most common debt is SHG-based microcredit. Women can borrow money from the SHG savings
pot (internal loans), borrowing rather small amounts averaging Rs 6,559. Or they take out SHG bank loans which are
larger, averaging nearly Rs 17,000, and are used by over a fifth of the sample. Loans from employers are also impor-
tant. Agricultural workers were historically entangled in debt-relations with their landlords—be it as pannaiyaal (per-
manent farm servant) or as lenders of first resort. While pannaiyaal has transformed significantly, landowners
continue to be an important credit source, with 24\% of households in our sample in debt to a landowner.5 Following
rural industrialization, debt bondage re-emerged within the rural powerloom industry through cash advances to
workers (Carswell \& De Neve, 2013). 

Good debt, by contrast, is debt that is cheap, or at least more affordable, not socially debasing and that
might even enhance one's status within the family or wider community. Good debt allows one to invest in the
future, through marriages, education or house building, and can be both status enhancing and productive in the
long run. As Guérin writes, unless debts become unmanageable, being in debt itself ‘is not a symptom of poor
management or financial illiteracy but a sign of responsibility’ (2014, p. S44). Indeed, rather than being frowned
upon, getting into debt can be morally accepted and even valued, such as when you borrow to contribute to
ceremonial exchanges (seeru), help with family emergencies, invest in education, or spend on children's mar-
riages. Women's borrowing to buy gold for their daughters, for example, is widely valued as an honourable and
responsible thing to do.


% ******************************
\section{Gift as profit}
% ******************************


\cite{Guerin2020c} : savings gift ceremonials
Proposent que le mariage lui même soit une forme d'épargne
Donc on investi beaucoup d'argent dedans (à moi de le montrer), mais c'est une sorte d'épargne

In the absence – or the dismantling – of public health and social security systems, bank saving is
expected to help the poor to cope with risk and emergencies, to better plan for the future, whether for
life cycle and ritual events, education, housing or economic investments (Collins et al. 2009).

People keep explicit accounts of contributions and
receipts, and switch between debtor and creditor positions across their life cycle, but accounting
remains imbued with social and emotional considerations.

Achievements in terms of savings deposit,
however, are less convincing. The median amount is unchanged (around 600 INR) and the average
amount had even decreased (from 4470 INR to 2043 INR). Most bank accounts are in fact ‘dormant’
and mostly used as a conduit for social transfers. 
In villages, we were frequently told that bank deposits were ‘useless.’ People were clear that infor-
mal saving made more sense, both socially and financially. Gold, in particular, meets a much higher
demand for storing value than keeping money in a bank account. As in 2010, gold is still the most
important form of saving. Most households own gold (95.7\%) at an average weight of 52.2 grams, for
an average value of 155,653 INR (76 times more than the average value of bank deposits).
"pour tacler \cite{Anukriti2020} si jamais elle parle d'épargne classique"

 In rural Tamil Nadu, the reappropriation of rituals, whether in terms of meaning or funding, is a powerful
tool for asserting individual and collective identities, especially among Dalits (De Neve 2000, Picherit
2018) At the same time, some dominant group rituals continue to spread. This is the case of the
dowry, a Brahmin practice which many social groups have adopted in recent years, and around
the 1960s in Tamil Nadu (Kapadia 1996). The prevalence of the dowry is often presented, perceived
and upheld, as pre-mortem compensation for non-access to inheritance for girls, including by the
women themselves. Although generous dowries certainly boost the social standing of households,
clans and the women themselves in their community (much less than it being material protection
for the women, as it is most often appropriated by in-laws), it is obviously a symptom of – and a
powerful tool of – entrenched patriarch

Families usually keep one notebook per event (see Figures 1 and 2 as examples), with a list of the
givers specifying their names, location and the amount of their gift, which can be in cash or in
kind, mostly gold, clothes, vessels and food.7 Gifts in kind are restricted to relatives (exceptionally
close friends may give gold).








Rao, V., 2001. Celebrations as social investments: festival expenditures, unit price variation and social status in rural
India. Journal of Development Studies, 38 (1), 71–97.




\cite{Heyer1992} : Dowry and savings

\cite{Goedecke2018} : Financial policy and savings

\cite{Anukriti2020} : finance dowry through savings and saving chanels






% ******************************
\section{Discussion and conclusion}
\label{section:conclusion}
%\addcontentsline{toc}{section}{Conclusion}
% ******************************











\newpage
%-------------------------------------------------------------------------------%
%\begin{nolinenumbers}
\addcontentsline{toc}{section}{Références}
%\bibliography{C:/Users/Arnaud/Dropbox/Arnaud/Ref_Arnaud}
\bibliography{C:/Users/anatal/Downloads/Dropbox/Arnaud/Ref_Arnaud}
%\nocite{*}

\clearpage
\newpage
%-------------------------------------------------------------------------------%
\setcounter{tocdepth}5
\tableofcontents

%\end{nolinenumbers}
\end{document}
\documentclass[a4paper, 11pt, onecolumn]{article} 

% arara: pdflatex 
% arara: bibtex
% arara: pdflatex
% arara: pdflatex
% arara: clean: {extensions: [ aux, bbl, out, toc, blg, thm ]}

\usepackage[doi, natbibapa]{apacite}

\usepackage{enumitem}
\usepackage[english]{babel}
%if french
%\frenchbsetup{StandardLists=true}

\usepackage[T1]{fontenc}
\usepackage[utf8]{inputenc}

\usepackage{lmodern}

\let\CheckCommand\providecommand
\usepackage{microtype}
\usepackage{hyperref}
%\usepackage[pagebackref=true]{hyperref}
%\renewcommand*{\backrefalt}[4]{#1}

\usepackage{lscape}
\usepackage{graphicx}
\usepackage{amssymb,amsmath}
\usepackage{url}
\usepackage{longtable}
\usepackage{tabu}
\usepackage{siunitx}                        
%\usepackage{threeparttable} 
\usepackage{array}
\usepackage{booktabs}

%\usepackage[french]{authblk}
%\DeclareCaptionFormat{twodot}{:}
\usepackage[font=small,skip=1em]{caption}


\usepackage{setspace}
\usepackage{fullpage}
\usepackage{eso-pic}

\usepackage[explicit, clearempty]{titlesec}
\usepackage[tableposition=top]{caption}
%\usepackage{titlesec}
\usepackage[a4paper]{geometry}

\usepackage{adjustbox}
\usepackage{rotating}
\usepackage{hvfloat}
\usepackage{wrapfig}
\usepackage{tfrupee}  
%\usepackage{multicol}

\usepackage{calc}

\usepackage{lettrine}
\usepackage{oldgerm}

\usepackage{fancyhdr}
\usepackage{lipsum}  
\usepackage{lastpage}

\usepackage{changepage}

% *****************************************************************
% Annexes
% *****************************************************************
%\usepackage[title, titletoc]{appendix}
\usepackage[toc,page]{appendix}
%\renewcommand\appendixtocname{Annexes}
%\renewcommand\appendixname{Annxes}
%\renewcommand\appendixpagename{Annexes}

% *****************************************************************
% Estout related things
% *****************************************************************
\newcommand{\sym}[1]{\rlap{#1}}

\let\estinput=\input% define a new input command so that we can still flatten the document

\newcommand{\estwide}[3]{
		\vspace{.75ex}{
			\begin{tabular*}
			{\textwidth}{@{\hskip\tabcolsep\extracolsep\fill}l*{#2}{#3}}
			\toprule
			\estinput{#1}
			\bottomrule
			\addlinespace[.75ex]
			\end{tabular*}
			}
		}	

\newcommand{\estauto}[3]{
		\vspace{.75ex}{
			\begin{tabular}{l*{#2}{#3}}
			\toprule
			\estinput{#1}
			\bottomrule
			\addlinespace[.75ex]
			\end{tabular}
			}
		}

% Allow line breaks with \\ in specialcells
	\newcommand{\specialcell}[2][c]{%
	\begin{tabular}[#1]{@{}c@{}}#2\end{tabular}}

% *****************************************************************
% Custom subcaptions
% *****************************************************************
% Note/Source/Text after Tables
\newcommand{\figtext}[1]{
	\vspace{-1.9ex}
	\captionsetup{justification=justified,font=footnotesize}
	\caption*{\hspace{6pt}\hangindent=1.5em #1}
	}
\newcommand{\fignote}[1]{\figtext{\emph{Note:~}~#1}}

\newcommand{\figsource}[1]{\figtext{\emph{Source:~}~#1}}

% Add significance note with \starnote
\newcommand{\starnote}{\figtext{* p < 0.1, ** p < 0.05, *** p < 0.01. Standard errors in parentheses.}}

% *****************************************************************
% siunitx
% *****************************************************************
\usepackage{siunitx} % centering in tables
	\sisetup{
		detect-mode,
		tight-spacing		= true,
		group-digits		= false ,
		input-signs		= ,
		input-symbols		= ( ) [ ] - + *,
		input-open-uncertainty	= ,
		input-close-uncertainty	= ,
		table-align-text-post	= false
        }

% *****************************************************************
% Sources
% *****************************************************************
\newcommand{\sourcetab}[1]{\vspace{-1em} \caption*{ \textbf{Source}: {#1}} }
\newcommand{\sourcefig}[1]{\vspace{-2em} \caption*{ \textbf{Source}: {#1}} }

\addto\captionsenglish{\renewcommand{\figurename}{\textbf{Figure}}}
\addto\captionsenglish{\renewcommand{\tablename}{\textbf{Table}}}


% *****************************************************************
% Abstract
% *****************************************************************
\let\abstractname\abstracteng

% *****************************************************************
% Hypothèses
% *****************************************************************
\usepackage{ntheorem}
\theoremseparator{:}
\newtheorem{hyp}{Hypothesis}

% \makeatletter
% \newcounter{subhyp} 
% \let\savedc@hyp\c@hyp
% \newenvironment{subhyp}
 % {%
  % \setcounter{subhyp}{0}%
  % \stepcounter{hyp}%
  % \edef\saved@hyp{\thehyp}% Save the current value of hyp
  % \let\c@hyp\c@subhyp     % Now hyp is subhyp
  % \renewcommand{\thehyp}{\saved@hyp\alph{hyp}}%
 % }
 % {}
% \newcommand{\normhyp}{%
  % \let\c@hyp\savedc@hyp % revert to the old one
  % \renewcommand\thehyp{\arabic{hyp}}%
% } 
% \makeatother

% *****************************************************************
% Tableaux
% *****************************************************************
\addto\captionsfrench{\def\tablename{\textsc{Table}}}

% *****************************************************************
% Bigcenter
% *****************************************************************
%%% ----------debut de bigcenter.sty--------------
 
%%% nouvel environnement bigcenter
%%% pour centrer sur toute la page (sans overfull)
 
%\newskip\@bigflushglue \@bigflushglue = -100pt plus 1fil
 
%\def\bigcenter{\trivlist \bigcentering\item\relax}
%\def\bigcentering{\let\\\@centercr\rightskip\@bigflushglue%
%\leftskip\@bigflushglue
%\parindent\z@\parfillskip\z@skip}
%\def\endbigcenter{\endtrivlist}
 
%%% ----------fin de bigcenter.sty--------------
%%%% fin macro %%%%
\makeatletter
\newskip\@bigflushglue \@bigflushglue = -100pt plus 1fil
\def\bigcenter{\trivlist \bigcentering\item\relax}
\def\bigcentering{\let\\\@centercr\rightskip\@bigflushglue%
\leftskip\@bigflushglue
\parindent\z@\parfillskip\z@skip}
\def\endbigcenter{\endtrivlist}
\makeatother

% *****************************************************************
% Lignes de code
% *****************************************************************
\usepackage{listings}
\lstset{ 
basicstyle=\scriptsize\ttfamily,
breaklines=true,
keywordstyle=\bf \color{blue},
commentstyle=\color[gray]{0.5},
stringstyle=\color{red},
showstringspaces=false,
numbers=left,
numberstyle=\tiny \bf \color{blue},
stepnumber=1,
numbersep=10pt,
firstnumber=1,
numberfirstline=true,
frame=leftline,
xleftmargin=0.5cm
}
 
% *****************************************************************
% Auteurs en bleu
% *****************************************************************
%\renewcommand{\citep}[1]{\textcolor{teal}{\citep{#1}}}
%\renewcommand{\cite}[1]{\textcolor{teal}{\cite{#1}}}
\usepackage{xcolor}
\usepackage{colortbl}

\hypersetup{colorlinks,linkcolor={red},citecolor={teal},urlcolor={blue}}
%\newcommand{\ypenser}[1]{\textcolor[purple]{#1}}
\newcommand{\ypenser}[1]{\textbf{\color{purple}--#1--}}

% *****************************************************************
% Mail
% *****************************************************************
\newcommand{\email}[1]{\href{mailto:#1}{\nolinkurl{#1}}}

% *****************************************************************
% Résumé et mots clés
% *****************************************************************
 \newenvironment{resab}[1]
{\begin{adjustwidth}{0cm}{0cm} \hangafter =1\par
    {\normalsize\bfseries #1\ \\ }\normalsize}
{\end{adjustwidth}\medskip}

 \newenvironment{keywords}
{\begin{adjustwidth}{0cm}{0cm} \hangafter =1\par
    {\normalsize\itshape Keywords:}~\normalsize}
{\end{adjustwidth}\medskip}

 \newenvironment{jelcodes}
{\begin{adjustwidth}{0cm}{0cm} \hangafter =1\par
    {\normalsize\itshape JEL Codes:}~\normalsize}
{\end{adjustwidth}\medskip}

% *****************************************************************
% Poete
% *****************************************************************
\newcommand{\attrib}[1]{%
\nopagebreak{\raggedleft\footnotesize #1\par}}

% *****************************************************************
% Stata
% *****************************************************************
\newcommand{\Stata}{%
\textsc{Stata$^{\mbox{\scriptsize{\textregistered}}}$}
}

% *****************************************************************
% Encadré
% *****************************************************************
\usepackage{tikz}

% Thick
\def\checkmark{\tikz\fill[scale=0.4](0,.35) -- (.25,0) -- (1,.7) -- (.25,.15) -- cycle;} 

\newcommand{\titlebox}[2]{%
\tikzstyle{titlebox}=[rectangle,inner sep=10pt,inner ysep=10pt,draw]%
\tikzstyle{title}=[fill=white]%
%
\bigskip\noindent\begin{tikzpicture}
\node[titlebox] (box){%
    \begin{minipage}{0.94\textwidth}
#2
    \end{minipage}
};
%\draw (box.north west)--(box.north east);
\node[title] at (box.north) {#1};
\end{tikzpicture}\bigskip%
}

% *****************************************************************
% Changer la forme des titres
% *****************************************************************
 %\titleformat{\section}[block]			%section + style prédéfini par l'extension (block = 1 ligne)
 %{\sffamily\bfseries\LARGE\titlerule[1pt]}						%format pour le titre + label
 %{\sffamily\bfseries\LARGE}						%format pour le titre + label
 %{\sffamily\bfseries\LARGE\arabic{section}}		%format que pour le label
 %{0.5cm}								%espace qui sépare le label du titre
 %{#1}									%description du style pour le titre uniquement
 %\titlespacing{\section}				%section
 %{0cm}									%espace à gauche du titre
 %{3em}									%espace verticale AVANT le titre
 %{1em}									%espace verticale APRES le titre
 %%{0cm} 								%espace à droite du titre: mettre la même valeur que gauche pour un peu centrer

 %\titleformat{\subsection}{\sffamily\bfseries\Large}{\thesubsection}{0.4cm}{#1}
 %\titlespacing{\subsection}
 %{1cm}
 %{2em}
 %{1ex}

 %\titleformat{\subsubsection}{\sffamily\bfseries\large}{\thesubsubsection}{0.4cm}{#1}
 %\titlespacing{\subsubsection}
 %{2cm}
 %{2em}
 %{1ex}

% *****************************************************************
% Style de la date
% *****************************************************************
%\def\mydate{\leavevmode\hbox{\the\year-\twodigits\month}}
\def\mydate{\leavevmode\hbox{\the\month-\twodigits\year}}
\def\twodigits#1{\ifnum#1<10 0\fi\the#1}

% *****************************************************************
% En tête
% *****************************************************************

% Pour les numéros de pages
%\pagestyle{fancy}

% Si tu veux mettre les numéros genre: 1/22
% Il faut que tu écrives
% "\thepage/\pageref{LastPage}"
% Sans les " " dans les lignes en bas: \fancyhead[...

% Ca c'est pour enlever la barre horizontale sous l'entête
% Pour la laisser tu mets "1pt" au lieu de 0
\renewcommand{\headrulewidth}{0pt}
\renewcommand{\footrulewidth}{0pt}

% fancyhead pour l'entête
% fancyfoot pour le pied de page
% L=left; R=right; C=center
%\fancyhead[L]{\textcolor{gray}{\textsf{\textit{Revue de la littérature autour du mariage: le cas de l'Inde}}}}
%\fancyhead[R]{\textcolor{gray}{\textsf{Natal, A. (\mydate)}}}
%\fancyfoot[C]{\thepage}
%updmap.exe --admin

%\lhead{\textcolor{gray}{\textsf{\textit{Revue de la littérature autour du mariage: le cas de l'Inde}}}}
%\rhead{\textcolor{gray}{\textsf{Natal, A. (\mydate)}}}
\cfoot{\thepage}

% *****************************************************************
% Jatis
% *****************************************************************
\newcommand{\jati}[1]{\textit{j\={a}ti{#1}}}

% *****************************************************************
% Enlever le titre Table des matières
% *****************************************************************
%\makeatletter
%\renewcommand\tableofcontents{%
%    \@starttoc{toc}%
%}
%\makeatother

% *****************************************************************
% À développer
% *****************************************************************
\newcommand\dev[1]{\textbf{\textcolor{red}{#1}}}

% *****************************************************************
% Fonts
% *****************************************************************
%\usepackage{tgbonum}
\usepackage{kpfonts}

% *****************************************************************
% Taille des tableaux
% *****************************************************************
\let\oldtabular=\tabular
\def\tabular{\small\oldtabular}
%\def\tabular{\normalsize\oldtabular}

% *****************************************************************
% Style de la biblio
% *****************************************************************
\bibliographystyle{apacite}

% *****************************************************************
% Numérotation
% *****************************************************************
%\usepackage{lineno}

% *****************************************************************
% Titre et page de garde
% *****************************************************************
\usepackage{titling}
\setlength{\droptitle}{-2cm}
\pretitle{\begin{center}\fontsize{24pt}{10pt}\selectfont\bfseries}
\posttitle{\par\end{center}\vskip 1ex}
\preauthor{\begin{center}
    \large \lineskip 0.5em}
\postauthor{\par\end{center}}
%\thanksheadextra{1,}{}
\thanksheadextra{}{}
\setlength\thanksmarkwidth{.5em}
\setlength\thanksmargin{-\thanksmarkwidth}


% *****************************************************************
% Symboles
% *****************************************************************

\def\@fnsymbol#1{\ensuremath{\ifcase#1\or *\or \dagger\or \ddagger\or
   \mathsection\or \mathparagraph\or \|\or **\or \dagger\dagger
   \or \ddagger\ddagger \else\@ctrerr\fi}}
   
\makeatletter
\newcommand{\ssymbol}[1]{^{\@fnsymbol{#1}}}
\makeatother
   
   
   
% *****************************************************************
% Makecell
% *****************************************************************
\usepackage{makecell}
\newcommand\Tablenote[2]{\multicolumn{#1}{l}{\makecell[l]{\textit{Note:}~#2}}}


   

%\usepackage{fonttable}

\usepackage{import}

\usepackage[
singlelinecheck=false % <-- important
]{caption}

\usepackage{upgreek}

\newcommand{\ie}{\textit{i.e.}}


\usepackage[most]{tcolorbox}
\newtcbtheorem{encadre}{Encadré}{enhanced, arc=0mm, interior style={white}, attach boxed title to top center= {yshift=-\tcboxedtitleheight/2}, breakable, fonttitle=\bfseries, fontupper=\itshape, colbacktitle=white,coltitle=black, boxed title style={size=normal, colframe=white, boxrule=0pt}}{enc}
\newtcbtheorem[auto counter]{greybox}{Box}{lower separated=false, breakable, colback=white!80!gray, colframe=white, fonttitle=\bfseries, colbacktitle=white!50!gray, coltitle=black, enhanced, boxed title style={colframe=black}, attach boxed title to top left={xshift=0.5cm,yshift=-2mm},}{encgris}
%\begin{#typedecadre}{#titre}{#label}
%\end{#typedecadre}



% *****************************************************************
% Interlignes et marges
% *****************************************************************
\setstretch{1}
\geometry{%
left=2.5cm,
right=2.5cm,
top=2.5cm,
bottom=2.5cm,
%includefoot,
%headsep=1cm,
%footskip=1cm
}%

% *****************************************************************
% Page de garde
% *****************************************************************
\title{Cost and Benefits of Marriage: Evidence from Rural India}
\author{Arnaud Natal\thanks{Univ. Bordeaux, CNRS, GREThA, UMR 5113, F-33600 \textsc{Pessac, France} - \email{arnaud.natal@u-bordeaux.fr}}}
\date{\today}
\renewcommand\maketitlehookc{%
  \begin{center}
    %\textsuperscript{1}{\small Université de Bordeaux}
    \textsuperscript{}{\normalsize Preliminary draft}
  \end{center}}

% ******************************************************************
\begin{document}
\maketitle

\hrule 
\vspace{0.3cm}

\begin{resab}{Abstract}

\end{resab}

\begin{motkey}{Keywords}
Marriage, dowry, rural India.
\end{motkey}


\hrule
%\linenumbers

% ******************************
\section*{Introduction}
\label{section:introduction}
%\addcontentsline{toc}{section}{Introduction}
\paragraph{Accroche}
\cite{Dubois1825} : en Inde, le mariage est la chose la plus importante




\paragraph{Problématisation}

Most of the literature on marriage in India deal with dowry and the consequences in terms of empowerment \citep{Roy2015, Alfano2017, Srinivasan2007} or the age at marriage and the child marriage \citep{Vogl2013, Sheela2003}
Few economist deal with the finance of marriage.
Recently, \cite{Anukriti2020} investigate the chanels to finance dowry in India.
They show that actual saving is drived by the expected amount of dowry [method].
\cite{Bloch2004} is one of the only research article that try to investigate the determinants of marriage expenses [literature].

\cite{Guerin2020c} show that ceremonials (including marriage) is a form of savings in rural India.
Married expect withdraw a profit from marriage with gift.


We try to understand the determinants of marriage benefits after 






\cite{Guerin2014a} : the problem is not debt, but to whom one becomes indebted




%-----------------------------------------------------
\section{Theoretical Framework}

\paragraph{Economics of marriage and family}
\cite{Becker1973} : base pour parler de l'économie des marriages



\paragraph{Marriage in south India}
\cite{Shulman1980, Meinzen1980} : déroulement du mariage hindoue en Inde du Sud 

\cite{Gupta1972, Gupta1976} : religiosity, economy and patterns of hindu marriage + love and arranged marriage

\cite{DeNeve2016} : economies of love in SI

\cite{Dharmalingam1994} : economics of marriage change in SI
\cite{Caldwell1983} : causes of marriage change in SI
\cite{Srinivasan2005} : dowry practice change in SI

\cite{Jejeebhoy2005} : économie des mariages en Inde du Sud
\cite{Reddy1991} : structure familiale et age au mariage en Inde du Sud
\cite{James2015} : stabilité de l'institution du mariage en Inde du Sud

\cite{Rao2001, Rao2001a}

\cite{Guerin2020c}
We attended the entire process for five of those, from the preparations, which involve
lively debates among family members and subtle calculations to determine the scale of the events, to
post-ceremonial remarks, both within the family (with heated debates on the generosity or conver-
sely the selfishness of the guests) and the neighbourhood (here with endless judgments on the greatness – or the smallness – of the event). Discussions about notebooks – a backbone of ceremonial
savings as we shall see below – were also key.7

First of all, these events are concrete oppor-
tunities to display and make visible the status (mariyātai) of the family. Ceremonies, insofar as they
gather the whole set of relations of individuals and their families, and insofar as they are in great part
funded by this same set of relations, express one’s mariyātai and contribute to building it. The scale of
the ceremony is evaluated in terms of guest numbers, their ‘quality,’ the quality of food provided, and
the gifts received: all this aims at maintaining, possibly uplifting or at least not downgrading the
organisers’ mariyātai (and that of their respective kin). Far from being simple, impersonal, individual
numbers of the kind found on a savings account statement, ceremonies reflect, crystalise and update
the social wealth of a group and its control over specific territories. These aspects are assessed in
terms of both the size of the local attendant population and the capacity to attract outsiders.

The scale of the ceremony mostly depends on two criteria. The scale of ceremonies recently
organised within the close circle (kinship and neighbourhood) is a first benchmark. Doing less
amounts to a downgrade in terms of honour. Doing better, even very slightly, is usually what is
expected.









\cite{Guerin2014a} : Census data indicates that Tamil Nadu is one of the states where household debt is the highest (NSSO
2003). 
Traditional” forms of rural debt based around extreme dependency between
landlords and labour are fading away (Cederlöf 1997; Guérin et al. 2009), as also observed in
other parts of India (Breman 1974; Breman et al. 2009). Labourers now have a wide range of
borrowing options. Empirical studies in the early 1980s highlighted the dynamism and
diversification of the rural financial landscape (Bouman 1989; Harriss B. 1981). In rural
Tamil Nadu for instance, professional lending, which had historically been the preserve of
specific castes, has opened up to other communities. Many local elites also used their cash
surpluses to invest as loans (Harriss B. 1981). More recent studies have shown that the
diversification of Tamil Nadu’s rural financial landscape is still going on (Ramachandran and
Swaminathan 2005; Polzin 2009).

Tamil Nadu is
moreover one of the states where microfinance has developed the most (Fouillet 2009).

De l'autre côté (du côté de la demande donc), il y a une envie de mobilité sociale = consumérisme, même en zone rurale Kapadia2002

Consumerism creates norms which many households are
willing to follow without having the financial means to do so (the “paradox of aspiration”
raised by Thorstein Veblen and also observed by Olsen and Morgan (2010) in Andhra
Pradesh). Households are borrowing on a daily basis at slack financial times to make ends
meet. They also borrow considerable amounts to marry their children, renovate their houses
or invest in private education. 
Avec RUME monthly debt service around half of their monthly income on average
D'autant plus graves que les revenus ne sont pas stables

On trouve trois types de ménages : 
transitional over indebtedness
pauperization
extreme dependance




\cite{Bloch2004} : 
To better understand the nature of marriage expenditures in India, it might help to
outline the basic nature of Indian marriage markets: Marriage is restricted to endoga-
mous groups. That is, people are only permitted to marry within a well-defined set of
families who make up their subcaste. A second feature of the marriage market is that
it is patrilocal4-brides leave their parents' home to live with their husbands. A third
is that marriages are arranged for both grooms and brides by their parents. Finally it
is important to note that marriage is considered final and, while there are cases of sep-
aration, divorce is not an option
D'autant plus en zones rurales





%-----------------------------------------------------
\section{Data and method}
\cite{NEEMSISreport}
\cite{NEEMSIS2017}

\subsection{Data}
Peut-être que je peux faire un matching avec pour les mariages de 2010-2016 comme ça je regarde avant marriage, juste après et au moins 4 ans après, même si ça me paraît difficile..!

Bien regarder la composition par sexe et par âge des ménages pour trouver les mêmes, mais juste qu'il n'y a pas eu de mariage.
Le mariage est la norme, donc en théorie, tous les HH qui ont une fille entre 18 et 25 ans ont vécu un mariage.
Je dois donc trouver des ménages avec des enfants non mariés qui ont l'age de ce marier.
Je dois faire un gros topos sur ça parce que c'est le plus important.


\cite{Guerin2014a} Difficultés à collecter données sur la dette car Many people under-estimate their debt levels for various reasons, including shame, concern for confidentiality,
fear of losing access (for fear that the survey results might be disclosed to NGOs or governmental units). For
further discussion of the challenges of collecting reliable data on debt, see Morvant-Roux (2006), and Collins et
al. (2009).
Morvant-Roux Solène, Processus d’appropriation des dispositifs de microfinance : un exemple en milieu
rural mexicain, Thèse de doctorat en sciences économiques, Université Lumière Lyon 2, 2006.

J'ai de la chance car excellente connaissance du terrain et présence de l'équipe depuis une vingtaine d'année donc confiance et fiabilité sont au rdv


\subsection{Method}
Bien regarder les prêts pour mariages
Montants, nombres, fréquences, intérêts, relations avec le prêteur, castes, 
Se fixer sur les choses importantes de \cite{Guerin2014a}




\cite{Bloch2004} : 
A large proportion of marriage costs are in the form of dowries.
To an outsider these weddings can seem extremely lavish, especially in contrast
with the extreme poverty of rural Indian life, with large numbers of people invited for
feasts and ceremonies that can go on for several days. Such celebrations are, of
course, not unique to India but are of special relevance in very poor societies where
the money spent on weddings can be particularly wasteful given its high opportunity
cost. 


While there is a large economics literature on marriage markets following the
work of Becker (1990), and a rapidly expanding literature on dowries,1 wedding cel-
ebrations are driven by different considerations and have not been addressed at all by
economists, or much by other social scientists.

 Our in-depth
interviews indicated that dowries were largely driven by competition for scarce men
and by the quality of grooms. However, marriage celebrations were driven by differ-
ent criteria that had more to do with symbolic display than transfers, with the bride's
family primarily responsible for paying for the wedding, just as they were responsi-
ble for paying the dowry. The size of the celebration was usually justified as being
forced by norms in the community that are usually determined by observing the scale
of other recent weddings in the community.

In addition, however, to what was considered a minimally acceptable wedding,
many families tended to have particularly lavish celebrations influenced less by norms
in the village than by patterns in cities with celebrations by poor families imitating the
more extravagant patterns common in richer families. 
Comme dans le papier de \cite{Guerin2014a} où les HH tendent vers les modes de consommations urbains bien qu'ils soient ruraux et très pauve

When asked why he
had spent so much money on a wedding that was obviously well beyond his means, the
father said that his daughter had married into a "good family" and he wanted to have a
"show." It should be noted that the son-in-law's father was a relatively wealthy
landowner from another village. Such clear distinctions between the motives driving
dowries, which our respondents said were about "purchasing" desirable grooms, and
wedding celebrations, which they indicated were of more symbolic value, were
expressed by most of the individuals with whom we had in-depth interviews.

Clearly, wedding celebrations have a lot to do with social status and prestige. What
does status mean in this context and how does it matter? Anthropologists have long
believed that Indian concepts of individuality differ markedly from the Western. An
Indian is defined not just by his or her own accomplishments and character, but also
by their circle of acquaintances and friends-how many important people they know,
and the status and respect accorded to them by their social group. Mines (1994), in a
study of a South India community, shows that men will often describe themselves to
a stranger not simply by providing information about who they are and what they do,
but by listing all their prominent acquaintances. In our own fieldwork, one village
leader described himself to us in a similar manner, "I am not a big man, but my father
was a freedom fighter and my daughter is married to a big family in Patilur village
(about 100 miles away). They have lots of land and her father-in-law is a big Congress
politician in the area." Thus, the village leader's sense of self seemed to be derived
from the "big men" to whom he was related.

Furthermore, mobility within a village is often achieved by imitating the behaviors
of families of higher social orders (Srinivas 1989). Families devote a great deal of
effort and expense to the presentation of external attributes. Household decisions are
often made with an emphasis on how one's family will be viewed by others: What will
others say? What will they think? For the parents of a daughter, marriage is potentially
the most important source of mobility because marrying into a "good family" can
greatly enhance how a family is viewed by its peers, and a prestigious match is an
occasion for great celebration and status displays. This, more than anything, explains
why some weddings are particularly lavish. Status is a value in itself.
While it may
also generate some secondary benefits like greater access to networks and informa-
tion, families clearly gain direct Utility from simply moving up the social ladder and
being associated by marriage with a prestigious, wealthy pedigree.3 Thus, when a
family marries into a rich family it is in their interest to demonstrate this to the rest of
the village, particularly if the rest of the village does not know the new in-laws. The
most effective way of signaling a family's newfound affinity-derived status is to have
as lavish a wedding as they can possibly afford. On the flip side, if a family marries a
poor local family-well
known to everyone in the village, this may also be an occa-
sion for celebration-but lavish displays are no longer necessary because not much
can be gained by signaling.

Marriage celebrations, however, follow a different pattern. They are large, averag-
ing about four months income, though much smaller than the dowry. In about 25 per-
cent of the cases, the bride's family reported spending nothing on marriage
celebrations, which is indicative of a small function restricted to immediate family
members or of rare cases (more common in the past) where the groom's family cele-
brated the wedding. Excluding these families, marriage expenses average over 5,000
rupees, which is about a third of the annual income of an average family. While the
celebration costs are small relative to the dowry, they still represent a major burden
for families at a time when substantial resources have to be spent on the dowry.

. When the bride's family is marrying into
a family from another village, however, the quality of the match is unknown to the
village of the bride's family and this change in status can be signaled to the village
by the size of the wedding celebration. This is suggested by the fact that in the sam-
ple analyzed in this paper, the average distance from the wife's home village for
families that practice village exogamy is 34 kilometers, which can be a considerable
distance in a society where roads and public transportation are very poor. In addi-
tion, while families themselves are quite careful about obtaining reasonably good
information on their prospective in-laws, the new in-laws are often quite unknown
to the other families in the village. Therefore, variations in village exogamy provide
a natural experiment that allows us to test if wedding celebrations are driven by sig-
naling motives

The processes by which dowries and marriage expenses are determined are some-
what different. Dowries are almost always the result of direct negotiations between
the families of the bride and the groom, though there is clear sense of what a reason-
able dowry is for any given match which is a function of the marriage market.
Wedding celebrations are also, to a degree, the result of negotiations between the
two families but to a much lesser extent than the dowry. A bride's family often spends
well beyond the level expected by the groom's family as a result of its own set of
incentives

Based on our ethnographic findings, we follow a "participatory econometric" \cite{Rao1997, Rao2002}.

Often dowries are not perfectly observable in the community, and hence cannot be used as perfect sig-
nals of the groom's quality. Clearly, rumors may circulate about the amount of the dowry paid, but the exact
amount of the dowry is not usually public information. There are at least two reasons why dowries are not
publicly observed. Firstly, they are illegal and hence are usually not publicly announced. Secondly, there are
clear incentive problems in the revelation of dowries. Bride's families have an incentive to underreport
the
amount paid, as high dowries may imply that the bride is of poor quality

Yh and Yw, the pre-marriage wealth of the two families, will be measured with
dummy variables for whether the families had any land.l4 Among these rural households, the possession of land makes a great deal of difference to a household's status
within the village because about 40 percent of the households are landless. To meas-
ure the characteristics of the husband and wife that are valued in the marriage market
xh and xw, we use their years of schooling and their ages at marriage. m the measure
of alternatives in the marriage market is measured by the year of marriage and the
marriage squeeze ratio \citep{Rao1993}-the ratio of the number of women to the number men at
marriageable ages (defined as women aged 10-19/ men aged 20-29). m is measured
at the year of the marriage for the district where the village is located, because the
census reports age tables by district. Also, because the census occurs every ten years
we use the ratio for the census year closest to the year of marriage. In addition to these
variables we also include dummies for whether the family is Muslim or belongs to
a scheduled caste, which are disadvantaged castes targeted by affirmative action
programs.
In the regression analysis we will examine a set of four specifications for each
dependent variable. The first will test Model 1, including only the wife's characteris-
tics, the second will include the husband's and wife's characteristics with the village
exogamy variable and no interactions, the third will add the interactions between vil-
lage exogamy and the wealth and education of the husband and wife. Finally, we
include a fourth regression introducing a new pair of variables, the average level of
education in the husband's village, and the interaction between village education and
exogamy. This is a measure of the opportunity cost for people from the husband's vil-
lage to attend the wedding. They attempt to answer a comment from a referee who
suggested that positive effects on interactions of wealthy husbands with exogamy may
simply indicate that husband's families with higher opportunity cost demand larger
outlays of expenditures during the wedding. If this is true, then the husbands from vil-
lages that are better educated should have more money spent on their wedding cele-
brations. After controlling for the average level of education in the husband's village
and its interaction with exogamy, the husband's own education and wealth should
measure the impact of the husbands own characteristics on wedding celebrations,
independent of the opportunity cost of his fellow villagers to attend the wedding.







%-----------------------------------------------------
\section{Burden of marriage}

\begin{greybox}{NEEMSIS 2 data at marriage level for description (n=117)}{marriageleveldesc}
\begin{itemize}[leftmargin=*]
\item 85.47\% of our marriages are arranged and only 6.84\% are consanguineous;
\item We find strong difference between caste for arranged marriage: 69.23\% for upper caste and 91.43 for middle;
\item In 85.47\% of marriage, the economic situation of husband or wife is the same than the observed husband or wife;
\item On average, the number of people at the wedding increase with the caste and on all sample, 372 persons are present at a marriage;
\item 90\% of marriage take place before lockdown and 8.55\% after;
\item We have 2 marriage during the lockdown;
\item 83.75\% of marriage take place in the same caste (dalits, middle, upper);
\item [...] thus 16.25\% (which represent 19 marriages) are inter caste\footnote{As \cite{Guerin2014a} stated, avec la dette normalement ils ne prêtent pas entre eux.} [literature] which is supported by the government;
\item 
\end{itemize}
\end{greybox}


\begin{greybox}{NEEMSIS 2 data at marriage level (n=117)}{marriagelevel}
\begin{itemize}[leftmargin=*]
\item Marriage cost, engagement cost and total cost increase with caste;
\item 50\% of marriage cost and engagement cost is provided by husband and 50\% by female for 50\% of marriages (44\% by wife for marriage on average and 74\% by wife for engagement on average);
\item Dowry represent 54\% of wife total cost (including engagement, marriage and dowry) but dowry represent 176\% of marriage and engagement fees;
\item Wife provide, on average, 66\% of total cost of \textbf{union (including engagement, marriage and dowry)} and the median is at 69\%;
\end{itemize}
\end{greybox}


\begin{greybox}{NEEMSIS 2 data at individual level (male=59; female=58)}{individuallevel}
\begin{itemize}[leftmargin=*]
\item In our sample, the marriage cost represent 47\% of the monetary value of assets;
\item [...], the engagement represent 11\% of the monetary value of assets;
\item For our sub sample of female, dowry represent 69\% of the monetary value of assets (the median is at 43\%);
\item Relative to the annual labor income of the household, the cost of marriage represent 85\% for male and 166\% for female;
\item The dowry represent 235\% of the annual labor income, around 2 years and 4 months of labor income;
\item The total cost of union represent 1 year of income for male on average and more than 3 years for female;
\item If we looked the expenses, they represent, on average, 85\% of the cost for male and 59\% of the marriage cost for female  (not including dowry, only marriage fees) which suppose that female need to borrow more than male to finance the marriage fees;
\item Surprisingly, when the husband/wife family is in better economic situation than the observed, the proportion of the payment is not higher than when the husband/wife family is in the same situation;
\item [...] Indeed, when the observed is a male (husband) and the wife's family is in better situation, the ratio of $\frac{marriage wife cost}{marriage total cost}$ equal 0.40 on average while it is at 0.46 on average when the wife family is in the same situation (for observed wife, the ratio is respectively 0.49 and 0.56);

\end{itemize}
\end{greybox}


\begin{greybox}{NEEMSIS 2 data for gift (male=59; female=58)}{gift}
\begin{itemize}[leftmargin=*]
\item 95\% of individuals (husband or wife of our sample) received gift; 
\item On average, individuals receive 133 000 rupees as gift for their marriage;
\item [...] that represent 38\% of the monetary value of assets and around one year of labor income (98\%);
\item Gift represent 138\% of the cost provided by the individual on average (the median is at 90\%);
\item Thus, 42\% of the individuals have a net profit on marriage fees (gift amount higher than cost);
\item This proportion is higher for wife than for husband because they pay less on marriage fees;
\item If we looked at the net benefits of marriage for wife ($benefits=gift-(fees+dowry)$) and husband ($benefits=(gift+dowry)-fees$), the average benefits is respectively at -266 120 rupees and +209 850 rupees;
\item [...] that represent a gain of 38\% of their assets for male (median) and a loss of 42\% for female (median);
\item in terms if labor income, it represent a gain of 11 months of labor income for husband family and a loss of 33 months for wife family;
\item Again with income, the median is at 0.61 for husband, which represent more than 7 months and the median is at 1.1 for wife, which represent more than 13 months;
\end{itemize}
\end{greybox}



\paragraph{Consumption}
\cite{Bloch2004} : Wedding celebrations as conspicuous consumption

\cite{Rosenzweig1989} : Marriage that smooth consumption

\cite{Guerin2014a} : Honouring reciprocity in ceremonies has always been a source of permanent pressure.
Many interviewees make clear that they prefer going into debt outside the family circle
to meet their own needs. This is a matter of freedom, as kin support calls for constant
justification (niyayapadthanum). Some say they borrow from their kin only for
"justified" reasons, which are mainly ceremony, housing and health costs. The
obligation of reciprocity (tiruppu) is also a burden. Not only should the debt be repaid,
but the debtor should be able to lend in return when the creditor is in need. 



\cite{Guerin2020c} Weddings are the most important events in terms of amount,
which vary according to social group. For Dalits, typical amounts in the region now range from three
to six lacks (300,000 to 600,000 INR), which on average amounts to four to eight years of household
income. These amounts have risen considerably over recent decades and include the cost of the cer-
ebration, gifts to close relatives and the dowry.

At the same time, one is supposed to organise an event within one’s financial and
human resources. Human resources are needed to help with the preparation of the food, service
and cleaning – the events usually bring together several hundred guests who must be welcomed, fed and sometimes accommodated. Financial resources depend on available savings (has the family
been able to accumulate gold, possibly land that can be offered, sold or pledged?), borrowing
capacities (how much can the family borrow and what is its creditworthiness in the eyes of potential
creditors?). Last but not least, how much can be expected from the guests in gifts (moi)? 
Le moi panam (la somme des dons offerts lors de l'événement) représente le plus souvent une part importante des dépenses de cérémonie, voire plus. En fait, mis à part les mariages de filles qui conduisent toujours à une «perte» due à la dot, chaque cérémonie devrait conduire à un surplus, un «profit», comme on nous l’a souvent dit (le terme anglais est fréquemment utilisé). En fait, le moi panam est le «nœud» de l’événement, son pilier. En d'autres termes, les cérémonies sont une sorte de pari sur la générosité de leur cercle social (à la fois en argent et en temps), qui à son tour dépend des obligations que les organisateurs ont accumulées au fil des ans. Pour les funérailles, les invités doivent fournir du riz et des articles de cuisine, qui doivent ensuite être retournés en temps opportun.

As we were told once, each event ‘is a link
to the past, the present and the future.’ It is an opportunity to show ‘the strength of your family
and relatives’ as we were also told, but the strength in question is a process: each event updates
prior relations and prefigures upcoming relations.


The event cost them around 1,50,000 INR10 (mostly food and transport, which was taken care of
for distant guests). They received 1,92,365 INR as gift (moi panam), which means a ‘surplus’ of
around 40,000 INR, that Sivaselvi used to pay off past debts. Gifts in cash ranged from 50 to 3000
INR. Gifts in gold ranged from a quarter of sovereign to 2.5 sovereigns (from around 5500 to
55,000 INR). The number and profile of the guests was a great source of satisfaction: most of the
neigbhours came, and many from outside and sometime long distances, revealing the territorial
spread of the family. In terms of gifts, however, Sivaselvi explains bitterly that she was expecting
a surplus of twice more.












%-----------------------------------------------------
\subsection{Burden of marriage debt}

Je dois revenir sur le fait que ce sont les plus pauvres qui s'endettent pour le plus pour les cérémonies. De plus, c'est pour cela que nous trouvons les montants de dettes les plus élevés.
Je devrais faire un panel avec les prêts pour montrer que les montants moyens de prêts pour mariage ne font que augmenter et que ce sont les castes les plus basses qui s'endettent avec ça.
Faire les ratios panel dette mariage / revenus annuel du travail pour montrer cette augmentation.


\cite{Guerin2011, Reboul2019} : Debt for why?

\cite{Agarwalla2015} : FInancial literacy among young

\citep{Carswell2021} : Marriage debt is a good debt

\cite{Guerin2014a} : social meaning of debt
While “financial inclusion” policies are now central to the political agendas of Indian public
policy makers (Garikipati 2008), private stakeholders such as NGOs and banks (Srinivasan
2009), and international organisations (World Bank 2007), this concern remains extremely
pressing

Over indebtedness and unmanageable can see as disaster : 
Dette --> suicide : Mohanty B. B. We are Like the Living Dead’: Farmer Suicides in Maharashtra, Western India, The Journal
of Peasant Studies, 32(2): 243–276, 2005

But it is also daily indebtedness: not necessarily lead to the dramatic situations observed amougst cotton farmers or microfinance clients but it can be nevertheless a source of impoverishment, pauperization and dependency \citep{Guerin2014a}

As for
the purposes for taking on debt, table 3 shows how in terms of debt size, the most significant
reasons include economic investment (mainly in agriculture) and ceremonies. In terms of the
number of loans, household expenses, economic investment and ceremonies are the most
common purposes. Here too significant disparities emerge as regards debt purpose: low castes, landless households and labourers more often borrow to cover daily survival costs and
ceremonies, while middle castes, landowners and producers more often borrow for economic
investment.

If we examine the main causes of over-indebtedness (see table 4 below), the most frequent are
ceremonies (42.65\% of households), housing and health (25\% and 23.53\%). These are
followed by failed economic investments (17.65\%); most frequently obtained for agricultural
purposes such as well digging or tractor purchase. 

When they are asked to talk about over-indebtedness, the borrowers rarely use
the amount of debt as an indicator. It is more the nature of the debt and the nature of the debt
relationship that determines whether debt is considered a burden or not.

Debt is a marker of social hierarchy in
kinship groups, the neighborhood and community alike. People try to avoid debts degrading
to their status, or at least try to pay back these debts first.

Borrowing from mobile lenders is seen as the most degrading practice, reserved for low castes
(and to women). Mobile lenders come to households’ doorsteps, precluding any form of
discretion. They do not request any collateral but use coercive enforcement methods. The
lenders themselves state that low caste individuals and women are more prepared to tolerate
abusive language from them.

The sense of debt as something immoral also depends upon the hierarchical positions of the
lender and the borrower. On the borrower’s side, the norm is to contract loans from someone
from an equal or higher caste." They do not take water from us, do you think they would take
money?" is something the low castes often said to us. On the creditor side, some upper castes
refuse to lend to castes who are too low in the hierarchy in comparison to them, stating that it
would be degrading for them to go and claim their due. To ask an upper caste whether he is
indebted to a lower caste can be considered as an insult.

The use of the term terinjavanga presupposes the idea of mutual
acquaintanceship (‘I know him/her, he/she knows me’)

While family support is crucial for
ceremonies and rituals

The most sensitive debts are those that do not respect the rules of rights and
obligations dictated by blood and alliances/bonds. For instance, borrowing money
from the bride’s kin is often a last resort, because it admits that the groom’s family is
unable to meet its responsibilities. Sometimes individuals may have no choice, but they
will be prompt in repayments, as the case study below illustrates.
Kathirvelu has an outstanding debt of 165 000 INR borrowed from twelve lenders, at rates ranging from 0 to
5\% per month. Of this, he considers only 28 000 INR as problematic. 20 000 INR is from pledging his
daughter-in-law’s jewellery. She allowed him to pledge 10 000 INR, but without her knowledge he
borrowed twice as much. The interest rate is rather low (1.5\% monthly) but this is not the problem: “I have
to keep face with the bride’s kin” he says. Another priority is a debt of 8000 INR from his son’s recruiter.
Here too, he does not want to show the bride’s kin that he is indebted. The family reputation is at stake: he
still has two daughters to marry, “we will never get them married if we have a reputation as indebted
family” (Kathirvelu, SC, agriculture coolie and tenant)

“Bad” debts are
rarely the most expensive, financially speaking, but those that tarnish the reputation of the
family and jeopardize its future, especially children’s marriages. Bad debts serve to reveal that a household is unable to maintain its position in the social hierarchy. The poor do
undertake financial reasoning, but financial criteria are not a priority and debt behaviours
stem from subtle arbitrations between financial costs and social status.
\cite{James2020} : debt, marriage and consumption










%-----------------------------------------------------
\section{Marriage gift as net benefits}
\cite{Guerin2020c} : savings gift ceremonials
Proposent que le mariage lui même soit une forme d'épargne
Donc on investi beaucoup d'argent dedans (à moi de le montrer), mais c'est une sorte d'épargne

In the absence – or the dismantling – of public health and social security systems, bank saving is
expected to help the poor to cope with risk and emergencies, to better plan for the future, whether for
life cycle and ritual events, education, housing or economic investments (Collins et al. 2009).

People keep explicit accounts of contributions and
receipts, and switch between debtor and creditor positions across their life cycle, but accounting
remains imbued with social and emotional considerations.

Achievements in terms of savings deposit,
however, are less convincing. The median amount is unchanged (around 600 INR) and the average
amount had even decreased (from 4470 INR to 2043 INR). Most bank accounts are in fact ‘dormant’
and mostly used as a conduit for social transfers. 
In villages, we were frequently told that bank deposits were ‘useless.’ People were clear that infor-
mal saving made more sense, both socially and financially. Gold, in particular, meets a much higher
demand for storing value than keeping money in a bank account. As in 2010, gold is still the most
important form of saving. Most households own gold (95.7\%) at an average weight of 52.2 grams, for
an average value of 155,653 INR (76 times more than the average value of bank deposits).
"pour tacler \cite{Anukriti2020} si jamais elle parle d'épargne classique"

 In rural Tamil Nadu, the reappropriation of rituals, whether in terms of meaning or funding, is a powerful
tool for asserting individual and collective identities, especially among Dalits (De Neve 2000, Picherit
2018) At the same time, some dominant group rituals continue to spread. This is the case of the
dowry, a Brahmin practice which many social groups have adopted in recent years, and around
the 1960s in Tamil Nadu (Kapadia 1996). The prevalence of the dowry is often presented, perceived
and upheld, as pre-mortem compensation for non-access to inheritance for girls, including by the
women themselves. Although generous dowries certainly boost the social standing of households,
clans and the women themselves in their community (much less than it being material protection
for the women, as it is most often appropriated by in-laws), it is obviously a symptom of – and a
powerful tool of – entrenched patriarch

Families usually keep one notebook per event (see Figures 1 and 2 as examples), with a list of the
givers specifying their names, location and the amount of their gift, which can be in cash or in
kind, mostly gold, clothes, vessels and food.7 Gifts in kind are restricted to relatives (exceptionally
close friends may give gold).








Rao, V., 2001. Celebrations as social investments: festival expenditures, unit price variation and social status in rural
India. Journal of Development Studies, 38 (1), 71–97.


\cite{Deolalikar1998b} : gender and savings in rural India

\cite{Heyer1992} : Dowry and savings

\cite{Goedecke2018} : Financial policy and savings

\cite{Anukriti2020} : finance dowry through savings and saving chanels







%-----------------------------------------------------
\section{Discussion}

\section*{Conclusion}
\label{section:conclusion}
%\addcontentsline{toc}{section}{Conclusion}












\newpage
%-------------------------------------------------------------------------------%
%\begin{nolinenumbers}
\addcontentsline{toc}{section}{Références}
\bibliography{C:/Users/Arnaud/Dropbox/Arnaud/Ref_Arnaud}
%\nocite{*}



\clearpage
\newpage
%-------------------------------------------------------------------------------%
\setcounter{tocdepth}5
\tableofcontents

%\end{nolinenumbers}
\end{document}
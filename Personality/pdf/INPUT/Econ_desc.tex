% Table generated by Excel2LaTeX from sheet 'Factor1-5'
\begin{table}[htbp]
  \raggedright
  \caption{Summary of specifications}
  \resizebox{\columnwidth}{!}{%
    \begin{tabular}{llccccc}
    \toprule
    Code  & Specifications & In debt (=1) & Loan amount & Number of loans & IDSR  & Over-indebtedness (=1) \\
    \midrule
    (1)   & All controls & \checkmark     & \checkmark     & \checkmark     & \checkmark     & \checkmark \\
    (2)   & + PTCS X Gender$\ssymbol{2}$ & \checkmark     & \checkmark     & \checkmark     & \checkmark     & \checkmark \\
    (3)   & + PTCS X Caste$\ssymbol{3}$ & \checkmark     & \checkmark     & \checkmark     & \checkmark     & \checkmark \\
    (4)   & + PTCS X Gender X Caste$\ssymbol{4}$ & \checkmark     & \checkmark     & \checkmark     & \checkmark     & \checkmark \\
    \midrule
    \multicolumn{2}{l}{Estimator} & Probit & OLS   & Poisson & OLS   & Probit \\
    \multicolumn{2}{l}{Interpretation} & M.E. & M.E.   & M.E. & M.E.   & M.E. \\
    \multicolumn{2}{l}{Number of individuals} & 835   & 606   & 606   & 606   & 606 \\
    \multicolumn{2}{l}{Description of individuals} & All egos   & All indebted egos   & All indebted egos   & All indebted egos   & All indebted egos \\	
    \bottomrule
	\Tablenote{7}{$\ssymbol{2}$Two-way interaction terms allow us to separate M.E. between sex, which mean that we obtains two columns: male and female. $\ssymbol{3}$Two-way \\ 
	 interaction terms allow us to separate M.E. between caste, which mean that we obtains two columns: dalits and middle-upper caste. $\ssymbol{4}$Three-way \\ 
	 interaction terms allow us to separate M.E. between gender and caste, which mean that we obtains four columns: muc male, dalits male, \\ 
	muc female and dalits female.} \\
    \end{tabular}%
	}
  \label{tab:econ_desc}%
  \sourcetab{NEEMSIS-1 (2016-17) and NEEMSIS-2 (2020-21).}
\end{table}%

\usepackage[doi, natbibapa]{apacite}

\usepackage{enumitem}
\usepackage[english]{babel}
%if french
%\frenchbsetup{StandardLists=true}

\usepackage[T1]{fontenc}
\usepackage[utf8]{inputenc}

\usepackage{lmodern}

\let\CheckCommand\providecommand
\usepackage{microtype}
\usepackage{hyperref}
%\usepackage[pagebackref=true]{hyperref}
%\renewcommand*{\backrefalt}[4]{#1}

\usepackage{lscape}
\usepackage{graphicx}
\usepackage{amssymb,amsmath}
\usepackage{url}
\usepackage{longtable}
\usepackage{siunitx}                        
\usepackage{threeparttable} 
\usepackage{array}
\usepackage{booktabs}

%\usepackage[french]{authblk}
%\DeclareCaptionFormat{twodot}{:}
\usepackage[font=small,skip=1em]{caption}


\usepackage{setspace}
\usepackage{fullpage}
\usepackage{eso-pic}

\usepackage[explicit, clearempty]{titlesec}
\usepackage[tableposition=top]{caption}
%\usepackage{titlesec}
\usepackage[a4paper]{geometry}

\usepackage{adjustbox}
\usepackage{rotating}
\usepackage{hvfloat}
\usepackage{wrapfig}
\usepackage{tfrupee}  
%\usepackage{multicol}

\usepackage{calc}

\usepackage{lettrine}
\usepackage{oldgerm}

\usepackage{fancyhdr}
\usepackage{lipsum}  
\usepackage{lastpage}

\usepackage{changepage}

% *****************************************************************
% Annexes
% *****************************************************************
%\usepackage[title, titletoc]{appendix}
\usepackage[toc,page]{appendix}
%\renewcommand\appendixtocname{Annexes}
%\renewcommand\appendixname{Annxes}
%\renewcommand\appendixpagename{Annexes}

% *****************************************************************
% Estout related things
% *****************************************************************
\newcommand{\sym}[1]{\rlap{#1}}

\let\estinput=\input% define a new input command so that we can still flatten the document

\newcommand{\estwide}[3]{
		\vspace{.75ex}{
			\begin{tabular*}
			{\textwidth}{@{\hskip\tabcolsep\extracolsep\fill}l*{#2}{#3}}
			\toprule
			\estinput{#1}
			\bottomrule
			\addlinespace[.75ex]
			\end{tabular*}
			}
		}	

\newcommand{\estauto}[3]{
		\vspace{.75ex}{
			\begin{tabular}{l*{#2}{#3}}
			\toprule
			\estinput{#1}
			\bottomrule
			\addlinespace[.75ex]
			\end{tabular}
			}
		}

% Allow line breaks with \\ in specialcells
	\newcommand{\specialcell}[2][c]{%
	\begin{tabular}[#1]{@{}c@{}}#2\end{tabular}}

% *****************************************************************
% Custom subcaptions
% *****************************************************************
% Note/Source/Text after Tables
\newcommand{\figtext}[1]{
	\vspace{-1.9ex}
	\captionsetup{justification=justified,font=footnotesize}
	\caption*{\hspace{6pt}\hangindent=1.5em #1}
	}
\newcommand{\fignote}[1]{\figtext{\emph{Note:~}~#1}}

\newcommand{\figsource}[1]{\figtext{\emph{Source:~}~#1}}

% Add significance note with \starnote
\newcommand{\starnote}{\figtext{* p < 0.1, ** p < 0.05, *** p < 0.01. Standard errors in parentheses.}}

% *****************************************************************
% siunitx
% *****************************************************************
\usepackage{siunitx} % centering in tables
	\sisetup{
		detect-mode,
		tight-spacing		= true,
		group-digits		= false ,
		input-signs		= ,
		input-symbols		= ( ) [ ] - + *,
		input-open-uncertainty	= ,
		input-close-uncertainty	= ,
		table-align-text-post	= false
        }

% *****************************************************************
% Sources
% *****************************************************************
\newcommand{\sourcetab}[1]{\vspace{-1em} \caption*{ \textbf{Source}: {#1}} }
\newcommand{\sourcefig}[1]{\vspace{-2em} \caption*{ \textbf{Source}: {#1}} }

\addto\captionsenglish{\renewcommand{\figurename}{\textbf{Figure}}}
\addto\captionsenglish{\renewcommand{\tablename}{\textbf{Table}}}


% *****************************************************************
% Abstract
% *****************************************************************
\let\abstractname\abstracteng

% *****************************************************************
% Hypothèses
% *****************************************************************
\usepackage{ntheorem}
\theoremseparator{:}
\newtheorem{hyp}{Hypothesis}

% \makeatletter
% \newcounter{subhyp} 
% \let\savedc@hyp\c@hyp
% \newenvironment{subhyp}
 % {%
  % \setcounter{subhyp}{0}%
  % \stepcounter{hyp}%
  % \edef\saved@hyp{\thehyp}% Save the current value of hyp
  % \let\c@hyp\c@subhyp     % Now hyp is subhyp
  % \renewcommand{\thehyp}{\saved@hyp\alph{hyp}}%
 % }
 % {}
% \newcommand{\normhyp}{%
  % \let\c@hyp\savedc@hyp % revert to the old one
  % \renewcommand\thehyp{\arabic{hyp}}%
% } 
% \makeatother

% *****************************************************************
% Tableaux
% *****************************************************************
\addto\captionsfrench{\def\tablename{\textsc{Table}}}

% *****************************************************************
% Bigcenter
% *****************************************************************
%%% ----------debut de bigcenter.sty--------------
 
%%% nouvel environnement bigcenter
%%% pour centrer sur toute la page (sans overfull)
 
%\newskip\@bigflushglue \@bigflushglue = -100pt plus 1fil
 
%\def\bigcenter{\trivlist \bigcentering\item\relax}
%\def\bigcentering{\let\\\@centercr\rightskip\@bigflushglue%
%\leftskip\@bigflushglue
%\parindent\z@\parfillskip\z@skip}
%\def\endbigcenter{\endtrivlist}
 
%%% ----------fin de bigcenter.sty--------------
%%%% fin macro %%%%
\makeatletter
\newskip\@bigflushglue \@bigflushglue = -100pt plus 1fil
\def\bigcenter{\trivlist \bigcentering\item\relax}
\def\bigcentering{\let\\\@centercr\rightskip\@bigflushglue%
\leftskip\@bigflushglue
\parindent\z@\parfillskip\z@skip}
\def\endbigcenter{\endtrivlist}
\makeatother

% *****************************************************************
% Lignes de code
% *****************************************************************
\usepackage{listings}
\lstset{ 
basicstyle=\scriptsize\ttfamily,
breaklines=true,
keywordstyle=\bf \color{blue},
commentstyle=\color[gray]{0.5},
stringstyle=\color{red},
showstringspaces=false,
numbers=left,
numberstyle=\tiny \bf \color{blue},
stepnumber=1,
numbersep=10pt,
firstnumber=1,
numberfirstline=true,
frame=leftline,
xleftmargin=0.5cm
}
 
% *****************************************************************
% Auteurs en bleu
% *****************************************************************
%\renewcommand{\citep}[1]{\textcolor{teal}{\citep{#1}}}
%\renewcommand{\cite}[1]{\textcolor{teal}{\cite{#1}}}
\usepackage{xcolor}

\hypersetup{colorlinks,linkcolor={red},citecolor={teal},urlcolor={blue}}
%\newcommand{\ypenser}[1]{\textcolor[purple]{#1}}
\newcommand{\ypenser}[1]{\textbf{\color{purple}--#1--}}

% *****************************************************************
% Mail
% *****************************************************************
\newcommand{\email}[1]{\href{mailto:#1}{\nolinkurl{#1}}}

% *****************************************************************
% Résumé et mots clés
% *****************************************************************
 \newenvironment{resab}[1]
{\begin{adjustwidth}{0cm}{0cm} \hangafter =1\par
    {\normalsize\bfseries #1\ \\ }\normalsize}
{\end{adjustwidth}\medskip}

 \newenvironment{motkey}[1]
{\begin{adjustwidth}{0cm}{0cm} \hangafter =1\par
    {\normalsize\itshape #1\ }:~\normalsize}
{\end{adjustwidth}\medskip}

% *****************************************************************
% Poete
% *****************************************************************
\newcommand{\attrib}[1]{%
\nopagebreak{\raggedleft\footnotesize #1\par}}

% *****************************************************************
% Stata
% *****************************************************************
\newcommand{\Stata}{%
\textsc{Stata$^{\mbox{\scriptsize{\textregistered}}}$}
}

% *****************************************************************
% Encadré
% *****************************************************************
\usepackage{tikz}
\newcommand{\titlebox}[2]{%
\tikzstyle{titlebox}=[rectangle,inner sep=10pt,inner ysep=10pt,draw]%
\tikzstyle{title}=[fill=white]%
%
\bigskip\noindent\begin{tikzpicture}
\node[titlebox] (box){%
    \begin{minipage}{0.94\textwidth}
#2
    \end{minipage}
};
%\draw (box.north west)--(box.north east);
\node[title] at (box.north) {#1};
\end{tikzpicture}\bigskip%
}

% *****************************************************************
% Changer la forme des titres
% *****************************************************************
 %\titleformat{\section}[block]			%section + style prédéfini par l'extension (block = 1 ligne)
 %{\sffamily\bfseries\LARGE\titlerule[1pt]}						%format pour le titre + label
 %{\sffamily\bfseries\LARGE}						%format pour le titre + label
 %{\sffamily\bfseries\LARGE\arabic{section}}		%format que pour le label
 %{0.5cm}								%espace qui sépare le label du titre
 %{#1}									%description du style pour le titre uniquement
 %\titlespacing{\section}				%section
 %{0cm}									%espace à gauche du titre
 %{3em}									%espace verticale AVANT le titre
 %{1em}									%espace verticale APRES le titre
 %%{0cm} 								%espace à droite du titre: mettre la même valeur que gauche pour un peu centrer

 %\titleformat{\subsection}{\sffamily\bfseries\Large}{\thesubsection}{0.4cm}{#1}
 %\titlespacing{\subsection}
 %{1cm}
 %{2em}
 %{1ex}

 %\titleformat{\subsubsection}{\sffamily\bfseries\large}{\thesubsubsection}{0.4cm}{#1}
 %\titlespacing{\subsubsection}
 %{2cm}
 %{2em}
 %{1ex}

% *****************************************************************
% Style de la date
% *****************************************************************
%\def\mydate{\leavevmode\hbox{\the\year-\twodigits\month}}
\def\mydate{\leavevmode\hbox{\the\month-\twodigits\year}}
\def\twodigits#1{\ifnum#1<10 0\fi\the#1}

% *****************************************************************
% En tête
% *****************************************************************

% Pour les numéros de pages
%\pagestyle{fancy}

% Si tu veux mettre les numéros genre: 1/22
% Il faut que tu écrives
% "\thepage/\pageref{LastPage}"
% Sans les " " dans les lignes en bas: \fancyhead[...

% Ca c'est pour enlever la barre horizontale sous l'entête
% Pour la laisser tu mets "1pt" au lieu de 0
\renewcommand{\headrulewidth}{0pt}
\renewcommand{\footrulewidth}{0pt}

% fancyhead pour l'entête
% fancyfoot pour le pied de page
% L=left; R=right; C=center
%\fancyhead[L]{\textcolor{gray}{\textsf{\textit{Revue de la littérature autour du mariage: le cas de l'Inde}}}}
%\fancyhead[R]{\textcolor{gray}{\textsf{Natal, A. (\mydate)}}}
%\fancyfoot[C]{\thepage}
%updmap.exe --admin

%\lhead{\textcolor{gray}{\textsf{\textit{Revue de la littérature autour du mariage: le cas de l'Inde}}}}
%\rhead{\textcolor{gray}{\textsf{Natal, A. (\mydate)}}}
\cfoot{\thepage}

% *****************************************************************
% Jatis
% *****************************************************************
\newcommand{\jati}[1]{\textit{j\={a}ti{#1}}}

% *****************************************************************
% Enlever le titre Table des matières
% *****************************************************************
%\makeatletter
%\renewcommand\tableofcontents{%
%    \@starttoc{toc}%
%}
%\makeatother

% *****************************************************************
% À développer
% *****************************************************************
\newcommand\dev[1]{\textbf{\textcolor{red}{#1}}}

% *****************************************************************
% Fonts
% *****************************************************************
%\usepackage{tgbonum}
\usepackage{kpfonts}

% *****************************************************************
% Taille des tableaux
% *****************************************************************
\let\oldtabular=\tabular
\def\tabular{\small\oldtabular}
%\def\tabular{\normalsize\oldtabular}

% *****************************************************************
% Style de la biblio
% *****************************************************************
\bibliographystyle{apacite}

% *****************************************************************
% Numérotation
% *****************************************************************
%\usepackage{lineno}

% *****************************************************************
% Titre et page de garde
% *****************************************************************
\usepackage{titling}
\setlength{\droptitle}{-2cm}
\pretitle{\begin{center}\fontsize{24pt}{10pt}\selectfont\bfseries}
\posttitle{\par\end{center}\vskip 1ex}
\preauthor{\begin{center}
    \large \lineskip 0.5em}
\postauthor{\par\end{center}}
%\thanksheadextra{1,}{}
\thanksheadextra{}{}
\setlength\thanksmarkwidth{.5em}
\setlength\thanksmargin{-\thanksmarkwidth}


% *****************************************************************
% Symboles
% *****************************************************************

\def\@fnsymbol#1{\ensuremath{\ifcase#1\or *\or \dagger\or \ddagger\or
   \mathsection\or \mathparagraph\or \|\or **\or \dagger\dagger
   \or \ddagger\ddagger \else\@ctrerr\fi}}
   
\makeatletter
\newcommand{\ssymbol}[1]{^{\@fnsymbol{#1}}}
\makeatother
   
   
   
% *****************************************************************
% Makecell
% *****************************************************************
\usepackage{makecell}
\newcommand\Tablenote[2]{\multicolumn{#1}{l}{\makecell[l]{\textit{Note:}~#2}}}


   
   
   
   
     
% *****************************************************************
% Cadre
% ***************************************************************** 
\newcommand{\ie}{\textit{i.e.}}
\usepackage[most]{tcolorbox}

\newtcbtheorem{encadre}{Encadré}{enhanced, arc=0mm, interior style={white}, attach boxed title to top center= {yshift=-\tcboxedtitleheight/2}, breakable, fonttitle=\bfseries, fontupper=\itshape, colbacktitle=white,coltitle=black, boxed title style={size=normal, colframe=white, boxrule=0pt}}{enc}

\newtcbtheorem[auto counter]{greybox}{Box}{lower separated=false, breakable, colback=white!80!gray, colframe=white, fonttitle=\bfseries, colbacktitle=white!50!gray, coltitle=black, enhanced, boxed title style={colframe=black}, attach boxed title to top left={xshift=0.5cm,yshift=-2mm},}{labox}
